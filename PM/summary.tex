\documentclass{article}

\usepackage{summary}

\subject{Projektmanagement}
\semester{Summer 2024}
\author{Leopold Lemmermann}

\begin{document}\createtitle
\section{Begriffe \& Zertifizierung}
\subsection{Projekt}
\begin{itemize}
  \item temporary
  \item unique
  \item interdisciplinary
  \item progressive elaboration
  \item limited resources
  \item goal
\end{itemize}

\subsection{Projektmanagement}
\begin{itemize}
  \item Als Aufgabenbereich: \begin{table}
  \centering
  \begin{tabular}{c|c}
    Klassisch                    & Agil                         \\
    \hline
    sequenzieller Phasenprozess  & iterativer Prozess           \\
    feste Anforderungen          & variable Anforderungen       \\
    kaum Stakeholder-Interaktion & viel Stakeholder-Interaktion \\
    Verantwortung bei Leitung    & selbstorganisiertes Team     \\
  \end{tabular}
  \caption{Klassisches vs. Agiles Projektmanagement}
\end{table}
  \item fachlich: eigene Disziplin mit speziellem Wissen, eigenen Methoden, Techniken und Tools
  \item inhaltlich: Führungsaufgabe, oft synonym mit Projektleitung
\end{itemize}

\subsection{Zertifizierungsmodelle}
\begin{itemize}
  \item \textbf{PRINCE2} Projects in controlled environments: UK-Standard, PRINCE manual, 7 Prinzipien+Themen+Prozesse
  \item \textbf{PMI} Project Management Institute: US-Standard, PMBOK-Guide, Performance Domains \& 12 Prinzipien
  \item \textbf{Scrum.org / Scrum Alliance}: Scrum Guide, Zertifizierung in Rollen
  \item \textbf{IPMA} International Project Management Association: europäischer Ursprung, 4 Level, Kompetenznachweis
  \item \textbf{DIN 69901 \& ISO 21500}: klassische Normen für Projektmanagement
\end{itemize}

\section{Klassisches Projektmanagement}
\begin{enumerate}
  \item \textbf{Initialisierung}:
        \par Analyse IST \& SOLL, Wirtschaftlichkeit/Risiko/Chancen, Stakeholderanalyse, Projektauftrag, Vision/Ziele, Teams/Rollen, Kick-Off
        \par zB. Projektantrag, Business Case, Projektsteckbrief/Canvas, Stakeholderanalyse, SWOT-Analyse, Wirtschaftlichkeitsprüfung
  \item \textbf{Planung}:
        \par Lösungsvarianten, Ablauf, Finanzierung/Ressourcen, Arbeitsorganisation, Risiko-/Qualitätsmanagement
        \par zB. Projektplan/GANTT, Zuteilung Verantwortliche/Rechte, Team-/Skillmatrix, IT Ausrüstung, Risikoanalyse, QM-Plan
  \item \textbf{Umsetzung}:
        \par Controlling/Monitoring, Kommunikation
        \par zB. Produkt/Projektinkremente/inhaltliche Ergebnisse, Testberichte, Soll-Ist-Analysen
  \item \textbf{Abschluss}:
        \par Projektauswertung, Abschlussbericht
        \par zB. Serienfreigabe/Abnahmeprotokoll, Projektbericht, Lessons learned
\end{enumerate}

\subsection{Initialisierung}
\begin{itemize}
  \item \textbf{Project Canvas}: Projektsteckbrief, 1 Seite, 9 Felder (Ziel, Stakeholder, Risiken, etc.)
  \item \textbf{Stakeholderanalyse}: Betroffenheit \& Einfluss evaluieren, Maßnahmen planen
        \begin{itemize}
          \item Hohe Betroffenheit/Hoher Einfluss: Permament betreuen
          \item Hohe Betroffenheit/Geringer Einfluss: Beobachten
          \item Geringe Betroffenheit/Hoher Einfluss: Zufrienenstellen
          \item Geringe Betroffenheit/Geringer Einfluss: Nicht Informieren
        \end{itemize}
  \item \textbf{magisches Projektdreieck}: Leistung, Zeit, Kosten
  \item \textbf{SMARTe Ziele}: Spezifisch, Messbar, Attraktiv, Realistisch, Terminiert
  \item \textbf{Projektauftrag} (auch Charter/Definition): kurzes Dokument mit Verweisen, meist Fließtext, unterschrieben von Auftraggeber \& -nehmer
\end{itemize}

\subsection{Planung}
\begin{itemize}
  \item \textbf{Projektstrukturplan} (PSP, engl. Work Breakdown Structure WBS): konsistente/vollständige Basispläne für Durchführung mit variablem Inhalt.
\end{itemize}

\end{document}