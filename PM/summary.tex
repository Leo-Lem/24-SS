\documentclass{article}

\usepackage{summary}

\subject{Projektmanagement}
\semester{Summer 2024}
\author{Leopold Lemmermann}

\begin{document}\createtitle
\section{Begriffe \& Zertifizierung}
\subsection{Projekt}
\begin{enumerate}
  \item temporär (temporary)
  \item einmalig (unique)
  \item zielorientiert (goal)
  \item interdisziplinär (interdisciplinary)
  \item fortschreitende Ausarbeitung (progressive elaboration)
  \item begrenzte Ressourcen (limited resources)
\end{enumerate}

\subsection{Projektmanagement}
\begin{itemize}
  \item Als Aufgabenbereich: \begin{table}
  \centering
  \begin{tabular}{c|c}
    Klassisch                    & Agil                         \\
    \hline
    sequenzieller Phasenprozess  & iterativer Prozess           \\
    feste Anforderungen          & variable Anforderungen       \\
    kaum Stakeholder-Interaktion & viel Stakeholder-Interaktion \\
    Verantwortung bei Leitung    & selbstorganisiertes Team     \\
  \end{tabular}
  \caption{Klassisches vs. Agiles Projektmanagement}
\end{table}
  \item fachlich: eigene Disziplin mit speziellem Wissen, eigenen Methoden, Techniken und Tools
  \item inhaltlich: Führungsaufgabe, oft synonym mit Projektleitung
\end{itemize}

\subsection{Zertifizierungsmodelle}
\begin{itemize}
  \item \textbf{PRINCE2} Projects in controlled environments: UK-Standard, PRINCE manual, 7 Prinzipien+Themen+Prozesse
  \item \textbf{PMI} Project Management Institute: US-Standard, PMBOK-Guide, Performance Domains \& 12 Prinzipien
  \item \textbf{Scrum.org / Scrum Alliance}: Scrum Guide, Zertifizierung in Rollen
  \item \textbf{IPMA} International Project Management Association: europäischer Ursprung, 4 Level, Kompetenznachweis
  \item \textbf{DIN 69901 \& ISO 21500}: klassische Normen für Projektmanagement
\end{itemize}

\section{Klassisches Projektmanagement}
\begin{enumerate}
  \item \textbf{Initialisierung}:
        \par Analyse IST \& SOLL, Wirtschaftlichkeit/Risiko/Chancen, Stakeholderanalyse, Projektauftrag, Vision/Ziele, Teams/Rollen, Kick-Off
        \par zB. Projektantrag, Business Case, Projektsteckbrief/Canvas, Stakeholderanalyse, SWOT-Analyse, Wirtschaftlichkeitsprüfung
  \item \textbf{Planung}:
        \par Lösungsvarianten, Ablauf, Finanzierung/Ressourcen, Arbeitsorganisation, Risiko-/Qualitätsmanagement
        \par zB. Projektplan/GANTT, Zuteilung Verantwortliche/Rechte, Team-/Skillmatrix, IT Ausrüstung, Risikoanalyse, QM-Plan
  \item \textbf{Umsetzung}:
        \par Controlling/Monitoring, Kommunikation
        \par zB. Produkt/Projektinkremente/inhaltliche Ergebnisse, Testberichte, Soll-Ist-Analysen
  \item \textbf{Abschluss}:
        \par Projektauswertung, Abschlussbericht
        \par zB. Serienfreigabe/Abnahmeprotokoll, Projektbericht, Lessons learned
\end{enumerate}

\subsection{Initialisierung}
\begin{itemize}
  \item \textbf{Project Canvas}: Projektsteckbrief, 1 Seite, 9 Felder (Ziel, Stakeholder, Risiken, etc.)
  \item \textbf{Stakeholderanalyse}: Betroffenheit \& Einfluss evaluieren, Maßnahmen planen
        \begin{itemize}
          \item Hohe Betroffenheit/Hoher Einfluss: Permament betreuen
          \item Hohe Betroffenheit/Geringer Einfluss: Beobachten
          \item Geringe Betroffenheit/Hoher Einfluss: Zufrienenstellen
          \item Geringe Betroffenheit/Geringer Einfluss: Nicht Informieren
        \end{itemize}
  \item \textbf{magisches Projektdreieck}: Leistung, Zeit, Kosten
  \item \textbf{SMARTe Ziele}: Spezifisch, Messbar, Attraktiv, Realistisch, Terminiert
  \item \textbf{Projektauftrag} (auch Charter/Definition): kurzes Dokument mit Verweisen, meist Fließtext, unterschrieben von Auftraggeber \& -nehmer
\end{itemize}

\subsection{Planung}
\begin{itemize}
  \item \textbf{Projektstrukturplan} (PSP, engl. Work Breakdown Structure WBS): konsistente/vollständige Basispläne für Durchführung mit variablem Inhalt.
  \item \textbf{Risiken}: Eintrittswahrscheinlichkeit vs Schadensausmaß, Risiko ist einschätzbar <-> Krise nicht
  \item \textbf{Meilensteinplanung}
\end{itemize}

\subsection{Umsetzung}
\begin{itemize}
  \item \textbf{Soll-Ist-Vergleich}: Termine/Zeit, Leistung, Kosten (Pyramide)
  \item \textbf{Reviews}: Was? Wann? Wer? Ziel?
  \item \textbf{Statusreports} (Ampelphasen): Verschlechterung, gleichbleibend, Verbesserung
  \item \textbf{Änderungsmanagement}: zur Abstimmung, Kostenkontrolle, Zurückverfolgung
\end{itemize}

\subsection{Abschluss}
\begin{itemize}
  \item \textbf{Projektbeurteilung}: inhaltlich (Zieleerreichung, Qualität, Wirtschaftlichkeit, Folgekontrollen) und beziehungstechnisch (Zusammenarbeit, Probleme, Verbesserungen)
  \item \textbf{Projektabschlusssitzung}: Nachfolgeorganisation, kritische Rückschau, Learnings, Würdigung/Entlastung der Projektmitglieder, Organisationsauflösung
  \item \textbf{Projektabschlussbericht}: Formalisierung der Punkte unter Beurteilung
\end{itemize}

\section{Agiles Projektmanagement}
\begin{quote}Sensing and responding to change.\end{quote}

\begin{itemize}
  \item \textbf{Mindset}: attitude towards learning, change, …
  \item \textbf{Values}: see below
  \item \textbf{Principles \& Behavior}: personal/direct communication, welcome changes, …
  \item \textbf{Practices \& Processes}: Retro, Daily Standup, …
  \item \textbf{Methods, Enabling IT \& Frameworks}: Scrum, Kanban, …
\end{itemize}

\subsection{Charakteristika agiler Methoden}
\begin{itemize}
  \item \textbf{Iterative}: kurze Zyklen, schnelle Ergebnisse
  \item \textbf{Incremental}: schrittweise, kontinuierliche Verbesserung
  \item \textbf{Self-organising}: Teamarbeit, Kommunikation, Selbstorganisation
  \item \textbf{Adaptive}: Anpassung an Veränderungen, Flexibilität
\end{itemize}

\subsection{Agile Manifesto}
\begin{itemize}
  \item \textbf{Individuals and interactions} over processes and tools
  \item \textbf{Working software} over comprehensive documentation
  \item \textbf{Customer collaboration} over contract negotiation
  \item \textbf{Responding to change} over following a plan
  \item \textbf{Twelve Principles}
        \begin{enumerate}
          \item Customer satisfaction through early and continuous delivery of valuable software
          \item Welcome changing requirements, even late in development
          \item Deliver working software frequently
          \item Business people and developers must work together daily
          \item Build projects around motivated individuals
          \item The most efficient and effective method of conveying information to and within a development team is face-to-face conversation
          \item Working software is the primary measure of progress
          \item Agile processes promote sustainable development
          \item Continuous attention to technical excellence and good design enhances agility
          \item Simplicity
          \item Self-organizing teams
          \item Regular reflections on how to become more effective
        \end{enumerate}
\end{itemize}

\subsection{Cynefin Framework}
\begin{itemize}
  \item \textbf{Clear} (tight constraint/no degrees of freedom): sense $\to$ categorise $\to$ respond (Best Practice)
  \item \textbf{Complicated} (governing constraint/tightly coupled): sense $\to$ analyse $\to$ respond (Good Practice)
  \item \textbf{Complex} (enabling constraint/loosely coupled): probe $\to$ sense $\to$ respond (Emergent Practice)
  \item \textbf{Chaotic} (lacking constraint/decoupled): act $\to$ sense $\to$ respond (Novel Practice)
\end{itemize}

\subsection{Scrum Framework}
\begin{quote}Leichtgewichtiges Framework, welches Menschen, Teams und Organisationen hilft, komplexe Probleme adaptiv zu lösen.\end{quote}

\begin{itemize}
  \item \textbf{Roles}: Product Owner, Scrum Master, Development Team
  \item \textbf{Events}: Sprint, Sprint Planning, Daily Scrum, Sprint Review, Sprint Retrospective
  \item \textbf{Artifacts}: Product Backlog, Sprint Backlog, Increment
\end{itemize}

\subsubsection{Scrum Roles}
\begin{itemize}
  \item \textbf{Product Owner}: Verantwortlich für den Wert des Produkts, Priorisierung, Klarheit, Akzeptanzkriterien
  \item \textbf{Scrum Master}: Verantwortlich für Scrum, Prozess, Team, Organisation, Coach, Moderator, Impediment-Remover
  \item \textbf{Development Team}: Selbstorganisiert, cross-funktional, 3-9 Personen, Verantwortlich für das Produkt
\end{itemize}

\subsubsection{Scrum Events}
\begin{itemize}
  \item \textbf{Sprint} (alle): Ziel, fixe Arbeitsphase (enthält alle Events), keine Änderungen die Sprintziel gefährden
  \item \textbf{Sprint Planning} (alle): max. 8h, Ergebnis: Sprint Backlog, DoD
  \item \textbf{Daily Scrum} (Developer): Planung für nächsten Tag, max. 15min, Stand-Up
  \item \textbf{Sprint Review} (alle+Stakeholder): max. 4h, Ergebnis: Feedback, nächste Schritte
  \item \textbf{Sprint Retrospective} (alle): max. 3h, Verbesserung, Ergebnis: Verbesserungsplan
\end{itemize}

\subsubsection{Scrum Artifacts}
\begin{itemize}
  \item \textbf{Product Backlog}: Priorisierte Anforderungen, Schätzungen, Akzeptanzkriterien
  \item \textbf{Sprint Backlog}: Ausgewählte Anforderungen, Schätzungen, Akzeptanzkriterien
  \item \textbf{Increment}: Potenziell auslieferbares Produkt, am Ende des Sprints
\end{itemize}

\subsubsection{Schätzen}
\begin{itemize}
  \item \textbf{Story Points}: relative Schätzung, z.B. Fibonacci-Reihe, 1, 2, 3, 5, 8, 13, 21, 34, 55, 89
  \item \textbf{Magic Estimation}: Referenz festlegen, jeder verteilt still, diskutieren
  \item \textbf{Planning Poker}: Schätzen im Team, Karten, Diskussion, Konsens
\end{itemize}

\subsubsection{User Stories}
\begin{quote}Als [Rolle] möchte ich [Ziel], um [Nutzen] (Alle Ws außer Wie).\end{quote}

\begin{itemize}
  \item \textbf{Card}: Titel, ID, Beschreibung
  \item \textbf{Conversation}: Diskussion, Details
  \item \textbf{Confirmation}: Akzeptanzkriterien
\end{itemize}

\subsubsection{5-Stufen Retrospektive}
\begin{enumerate}
  \item \textbf{Set the stage}: Einstieg, Fokus, Ziel
  \item \textbf{Gather data}: Daten sammeln, Fakten, Meinungen
  \item \textbf{Generate insights}: Erkenntnisse, Ursachen, Lösungen
  \item \textbf{Decide what to do}: Maßnahmen, Verbesserungen
  \item \textbf{Close the retrospective}: Abschluss, Feedback, Ausblick
\end{enumerate}

\subsection{Agiles Mindset}

\begin{itemize}
  \item \textbf{Lernorientierung}: nach Lernmöglichkeiten suchen, mit neuen Dingen umgehen
  \item \textbf{Kollaborativer Austausch}: transparent/offen sein, gemeinsame Problembewältigung
  \item \textbf{Kunden Co-Creation}: Kunden in Entwicklung einbeziehen
  \item \textbf{Selbstführung}: Freiheitsgrade nutzen, selbst organisieren, Verantwortung übernehmen
\end{itemize}

\subsubsection{Rahmenbedingungen}

\begin{itemize}
  \item \textbf{Wissensimpulse}: Lernen, Weiterbildung, Wissenstransfer
  \item \textbf{Arbeitsgestaltung}: Arbeitsumgebung, Arbeitsmittel, Arbeitszeit
  \item \textbf{Führung}: Führungsverhalten, Führungskultur, Führungskompetenz
\end{itemize}

\end{document}