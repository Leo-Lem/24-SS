\documentclass{article}

\usepackage{xrcise}

\subject{Projektmanagement}
\semester{Summer 2024}
\author{Leopold Lemmermann}

\begin{document}\createtitle

\sheet[2024]{Altklausur WS (A)}
\begin{exercise}{Klassischer Teil}
  \begin{enumerate}
    \item Was ist ein Projekt?
    \item Erkläre und skizziere das empirische Vorgehen von Scrum?
    \item Skizziere und vergleiche Scrum mit Wasserfallprojekten
    \item Unterschied zwischen absoluten und relativem Schätzen erklären und warum.
  \end{enumerate}
\end{exercise}

\begin{exercise}{Agiler Teil}
  \begin{enumerate}
    \item Erkläre und nenne die 3C's einer User Story.
    \item Projektbeschreibung gegeben:
          \begin{enumerate}
            \item 3 User Stories mit jeweils 3 Akzeptanzkritierien erstellen
            \item Problem bei Projekt, Developer interagieren nicht mit Stakeholdern, Zombie Scrum. 8 Schritte von Overeem\&Veriwjs nennen, um dies zu beheben und wie sollte man diese aufs Projekt bezogen anwenden?
          \end{enumerate}
    \item Scrum Team erklären, eine Accountability von den 3 genauer erläutern
    \item 5 Schritte eine Retrospektive, nach 8 genannten Punkten
    \item Sinn Retroperspektive
    \item Cynefin erklären mit Skizze, was können Probleme sein,
  \end{enumerate}
\end{exercise}

\sheet[2024]{Altklausur SS (B)}
\begin{exercise}{Klassischer Teil}
  \begin{enumerate}
    \item Was ist ein Projekt?
    \item Unterschied zwischen agilen und klassischen Methoden erklären
    \item Alle 4 Phasen des klassischen pm erklären und jeweils 2 Ergebnisse aufschreiben
    \item Unterschied zwischen Definition of Done und akzeptanzkriterien
  \end{enumerate}
\end{exercise}

\begin{exercise}{Agiler Teil}
  \begin{enumerate}
    \item Erklären was eine Retro ist und warum sie wichtig ist und was das Ergebnis ist
    \item Beschreiben sie ein Scrum team und dessen Accountabilities und beschrieben sie eine Accountability genauer
    \item Beispielsitustion das Führungskräfte einfach Sachen in SB packen und wir mussten erklären was wir in den 5 Phasen der Retro machen mussten
    \item Warum sollten wir PBI schätzen und warum ist das wichtig
    \item Unsere Sprints machen keinen Wert wie überkommen wir Zombie-Scrum anhand eines Beispiels und wir sollten uns an den 8 Schritten orientieren.
  \end{enumerate}
\end{exercise}

\sheet[2023]{Altklausur SS}
\begin{exercise}{Theoretisches Wissen}
  \begin{enumerate}
    \item Gibt eine sinnvolle Definition für "Projekt" an.
    \item Zeichne die Kernprozesse als Schaubild auf. Außerdem sollen die einzelnen Kernprozesse stichpunktartig erklärt und die Abhängigkeiten klar werden.
    \item Welche drei Fragen werden in Daily Scrum beantwortet? Warum sind diese Informationen wichtig?
    \item Erkläre die Begriffe Epic, Task, Theme und User Story. Stelle dabei klar, wie sie jeweils zusammenhängen.
    \item Erläutere die Unterschiede zwischen iterativen und inkrementellen Verfahren. Nenne, wo diese zur Anwendung kommen.
    \item Erläutere das "Inspect \& Adapt" Verfahren. Nenne zwei Beispiele, wie und wo er eingesetzt wird.
  \end{enumerate}
\end{exercise}

\begin{exercise}{Prozessplanung}
  Das Studierendenwerk möchte eine Verwaltungssoftware für die Mensen haben. Dieses beinhaltet eine Planungskomponente, in der die Mensachefs die Speisepläne erstellen können aber auch Zugriff auf eine zentrale Rezeptdatenbank haben. Die Ansichtskomponente liefert eine Webanwendung und eine Schnittstelle für Bildschirme, um Wochen- und Tagespläne darstellen zu können. Die Bilanzierungskomponente stelle ein Kosten-Nutzen-Verhältnis dar und wertet das Nutzerverhalten aus. Für die Bearbeitung der Aufgaben sind wir ein Projektmanager, der für die Erstellung der Planungskomponente zuständig ist.
  Insgesamt wurden dann verschieden Anforderungen aufgelistet, die in 3 Releases aufgeteilt waren.

  \begin{enumerate}
    \item Zeichne ein PSP zu dem gegebenen Szenario, das anhand von Fachlichkeiten eingeteilt ist.
          \hint{Das PSP braucht 3 Ebenen (Ebene 0 zählt nicht mit). Pro Ebene müssen mindestens 3 Knoten eingetragen werden. Getroffene Annahmen sollen notiert werden.}
    \item Stelle einen Meilensteinplan auf.
  \end{enumerate}
\end{exercise}

\begin{exercise}{Agil}
  Gehe für diese Aufgabe davon aus, dass ihr in der zweiten Woche des ersten Scrum-Sprints seid.
  \begin{enumerate}
    \item Stelle ein Scrum-Planungs-Board auf.
    \item Trage in das Scrum-Planungs-Board 3 Epics ein. Ein Titel genügt.
    \item Zerlege eines der Epics in 3 User Storys.
    \item Formuliere zu einer User Story 3 Akzeptanzkriterien.
  \end{enumerate}
\end{exercise}

\end{document}