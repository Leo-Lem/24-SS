\documentclass{article}

\usepackage{xrcise}

\subject{Projektmanagement}
\semester{Summer 2024}
\author{Leopold Lemmermann}

\begin{document}\createtitle

% WS 2024
% 1. Was ist ein Projekt?
% 2. Erkläre und skizziere das empirische Vorgehen von Scrum?
% 3. Skizziere und vergleiche Scrum mit Wasserfallprojekten
% 4. Unterschied zwischen absoluten und relativem Schätzen erklären und warum man
% das relative bei komplexen Problemen anwenden sollte
% 5. Erkläre und nenne die 3C’s einer User Story.
% 6. Projektbeschreibung gegeben:
% a. 3 User Stories mit jeweils 3 Akzeptanzkritierien erstellen
% b. Problem bei Projekt, Developer interagieren nicht mit Stakeholdern, Zombie
% Scrum. 8 Schritte von Overeem&Veriwjs nennen, um dies zu beheben und
% wie sollte man diese aufs Projekt bezogen anwenden?
% 7. Scrum Team erklären, eine Accountability von den 3 genauer erläutern
% 8. 5 Schritte eine Retrospektive, nach 8 genannten Punkten
% 9. Sinn Retroperspektive
% 10. Cynefin erklären mit Skizze, was können Probleme sein, wenn man komplexe
% Projekte mit klassischen Projektmanagement angeht
% Klausur B:
% Klassischer Teil:
% 1. was ist ein Projekt
% 2. Unterschied zwischen agilen und klassischen Methoden erklären
% 3. Alle 4 Phasen des klassischen pm erklären und jeweils 2 Ergebnisse aufschreiben
% 4. Unterschied zwischen Definition of Done und akzeptanzkriterien
% Agiler Teil:
% 1. erklären was eine Retro ist und warum sie wichtig ist und was das Ergebnis ist
% 2. Beschreiben sie ein Scrum team und dessen Accountabilities und beschrieben sie eine
% Accountabiliy genauer
% 3. Beispielsitustion das Führungskräfte einfach Sachen in SB packen und wir mussten erklären
% was wir in den 5 Phasen der Retro machen mussten
% 4. Warum sollten wir PBI schätzen und warum ist das wichtig
% 5. Unsere Sprints machen keinen Wert wie überkommen wir zombi Scrum anhand eines
% Beispiels und wir sollten uns an den 8 Schritten orientieren

% SS2023
% 1. theoretisches Wissen (28 Punkte)
% 1.1: Gibt eine sinnvoll Definition für "Projekt" an.

% 1.2: Zeichne die Kernprozesse als Schaubild auf. Außerdem sollen die eizelnen Kernprozesse stichpunktartig erklärt und die Abhängigkeiten klar werden.

% 1.3: Welche drei Fragen werden in Daily Scrum beantwortet? Warum sind diese Informationen wichtig?

% 1.4: Erkläre die Begriffe Epic, Task, Theme und User Story. Stelle dabei klar, wie sie jeweils zusammenhängen.

% 1.5: Erläutere die Unterschiede zwischen iterativen und inkremetellen Verfahren. Nenne, wo diese zur Anwendung kommen.

% 1.6: Erläutere das "Inspect&Adept" Verfahren. Nenne zwei Beispiele, wie und wo er eingesetzt wird.



% Szenario
% Hier wurde ein Szenario erklärt. Diese Szenario soll dann für die folgenden beiden Aufgaben genutzt werden.

% Das Studierendenwerk möchte eine Verwaltungssoftware für die Mensen haben. Dieses beinhaltet eine Planungskomponente, in der die Mensachefs die Speisepläne erstellen können aber auch Zugriff auf eine zentrale Rezeptdatenbank haben. Die Ansichtskomponente liefert eine Webanwedung und eine Schnittstelle für Bildschirme, um Wochen- und Tagespläne darstellen zu können. Die Bilanzierungskomponente stelle ein Kosten-Nutzen-Verhältnis dar und wertet das Nutzerverhalten aus. Für die Bearbeitung der Aufgaben sind wir ein Projektmanager, der für die Erstellung der Planungskomponente zuständig ist.

% Insgesamt wurden dann verschieden Anforderungen aufgelistet, die in 3 Releases aufgeteilt waren.



% 2. Prozessplanung (20 Punkte)
% 2.1: Zeichne ein PSP zu dem gegebene Szenario, das anhand von Fachlichkeiten eingeteilt ist.

% Das PSP braucht 3 Ebenen (Ebene 0 zählt nicht mit)
% Pro Ebene müsssen mindestens 3 Knoten eingetragen werden
% Getroffene Annahmen sollen notiert werden
% 2.2: Stelle einen Meilensteinplan auf.

% Die Meilensteinen sollen das Format "24.12.2050, intern, Projektstart" haben.
% Erkläre für je 3 interen und 3 externe Meilensteine, warum diese intern bzw. extern sein sollen.
% Getroffene Annahmen sollen notiert werden.
% 3. Agil (12 Punkte)
% Gehe für diese Aufgabe davon aus, dass ihr in der zweiten Woche des ersten Scrum-Sprints seid.
% 3.1: Stelle ein Scrum-Planungs-Board auf.

% 3:2: Trage in das Scrum-Planungs-Board 3 Epics ein. Ein Titel genügt.

% 3.3: Zerlege eines der Epics in 3 User Storys.

% 3.4: Formuliere zu einer User Story 3 Akzeptanzkriterien.

\end{document}