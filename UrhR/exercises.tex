\documentclass{article}

\usepackage[solutions]{xrcise}

\subject{Urheberrecht in der Informationsgesellschaft}
\semester{Summer 2024}
\author{Leopold Lemmermann}

\begin{document}\createtitle

\sheet{SVS Fragenkatalog}
\setcounter{section}{20}
\begin{exercise}[3]{Spread Spectrum Watermarking}
  Gegeben sei ein Original (bereits frequenztransformiert)
  \[
    N(x,y)=\begin{pmatrix} 8 & 6 & 4 \\ 5 & 3 & 1\\ 6 & 2 & 0 \end{pmatrix}.
  \]
  Die Basisfunktionen $\Phi_i$ seien
  \[
    \Phi_1=\begin{pmatrix} 0 & 1 & 0 \\ 1 & 1 & 0 \\ 0 & 1 & 1 \end{pmatrix} \quad \text{und} \quad \Phi_2=\begin{pmatrix} 1 & 0 & 1 \\ 1 & 0 & 0 \\ 1 & 0 & 1 \end{pmatrix}.
  \]
  Es soll ein Watermark $b$ aus zwei Bit $b = ("0","1")$ eingebettet werden.

  \begin{enumerate}
    \item Berechnen Sie $D(x,y)$ (markiertes Original).
          \hint{Die $b_i$ werden transformiert in Werte aus $\{-1,1\}$, d.h. ein Null-Bit wird auf $-1$ abgebildet, ein Eins-Bit auf $1$.}
    \item Durch eine Störung (bzw. einen Angriff) sei die dritte Zeile von $D(x,y)$ ausgelöscht (mit Nullen belegt), d.h.
          \[
            \tilde{D}(x,y)=\begin{pmatrix} \cdot & \cdot & \cdot \\ \cdot & \cdot & \cdot \\ 0 & 0 & 0 \end{pmatrix}.
          \]
          Extrahieren Sie das Watermark. Als Schwellwert wird der Mittelwert der $o_i$ verwendet.
  \end{enumerate}

  \begin{solution}
    \begin{enumerate}
      \item Das markierte Original ergibt sich als $D(x,y) = N(x,y) + S(x,y)$. Da wir $N(x,y)$ bereits kennen, müssen wir lediglich $S(x,y)$ berechnen:

            $S(x,y) = \sum_i{b_i \cdot \Phi_i(x,y)} = \Phi_1(x,y) - \Phi_2(x,y) = \begin{pmatrix} 1 & -1 & 1 \\ 0 & -1 & 0 \\ 1 & -1 & 0 \end{pmatrix}$.

            Damit ist $D(x,y) = \begin{pmatrix} 9 & 5 & 5 \\ 5 & 2 & 0 \\ 7 & 1 & 0 \end{pmatrix}$.

      \item $o_i$ lässt sich berechnen als $o_i = \sum_{x,y}{D(x,y) \cdot \Phi_i(x,y)}$: $o_1 = 5+5+2 = 12$ und $o_2 = 9+5+5 = 19$.

            Der Schwellwert ist also $(12+19)/2 = 15.5$. Daraus ergibt sich das Watermark als $b = (0,1)$.
    \end{enumerate}
  \end{solution}
\end{exercise}

\setcounter{section}{25}
\begin{exercise}[1]{Inhalte auf Datenträgern}
  Nehmen wir an, dass von jedem Datenträger über kurz oder lang problemlos digitale Kopien der Daten angefertigt werden können.
  \begin{enumerate}
    \item Überlegen Sie sich verschiedene Möglichkeiten, wie ein Schutzsystem aussehen könnte, das die Nutzung der Inhalte nur Berechtigten (z.B. denen, die dafür bezahlt haben) erlaubt.
    \item Inwieweit unterscheidet sich die Problemstellung und Ihre Lösung(en) vom Schutz vor Raubkopien in der Software-Branche?
  \end{enumerate}

  \begin{solution}
    \begin{enumerate}
      \item Es gibt verschiedene Möglichkeiten, um ein Schutzsystem zu implementieren, das sicherstellt, dass nur Berechtigte Zugang zu den Inhalten haben:
            \begin{itemize}
              \item Digitale Rechteverwaltung (DRM): Hierbei werden die Inhalte verschlüsselt und können nur mit einem speziellen Schlüssel, der an die Lizenz des Benutzers gebunden ist, entschlüsselt werden. Dies könnte beispielsweise über eine Online-Authentifizierung geschehen, bei der der Benutzer sich bei jedem Zugriff auf den Inhalt verifizieren muss.
              \item Hardwaregebundene Lizenzen: Die Lizenz für den Zugriff auf die Daten könnte an eine bestimmte Hardware gebunden werden, wie z.B. an die Seriennummer eines Computers oder eines speziellen Dongles, der an den Computer angeschlossen werden muss.
              \item Cloud-basierte Lösungen: Anstatt die Daten physisch auf einem Datenträger zu speichern, könnten die Inhalte in der Cloud gespeichert und nur nach erfolgreicher Authentifizierung gestreamt werden. Der Zugang könnte an ein Benutzerkonto gebunden sein.
              \item Wasserzeichen: Digitale Wasserzeichen können in die Dateien eingefügt werden, um die Nachverfolgung von illegal verbreiteten Kopien zu ermöglichen. Diese Wasserzeichen könnten Informationen über den ursprünglichen Käufer enthalten.
            \end{itemize}
      \item Die Problemstellung unterscheidet sich in einigen wesentlichen Punkten vom Schutz vor Raubkopien in der Software-Branche:
            \begin{itemize}
              \item Datenformate und Nutzung: Während Software in der Regel ausgeführt wird, werden Inhalte auf Datenträgern häufig konsumiert (z.B. Videos, Musik, Bücher). Der Schutzmechanismus muss also nicht nur die Ausführung, sondern auch das Abspielen oder Lesen der Inhalte kontrollieren.
              \item Verschlüsselungsstrategien: Software-Raubkopien betreffen oft das Umgehen von Aktivierungsmechanismen, während bei digitalen Inhalten auch der Schutz vor der Entschlüsselung und illegalen Verbreitung eine größere Rolle spielt.
              \item Verteilung und Zugriff: In der Software-Branche gibt es oft einmalige Installationen, während bei digitalen Inhalten kontinuierlicher Zugriff erforderlich ist. Dies erfordert möglicherweise regelmäßige Überprüfungen und Authentifizierungen.
              \item Nutzerfreundlichkeit: Schutzmechanismen für digitale Inhalte müssen besonders benutzerfreundlich sein, um legale Nutzer nicht zu frustrieren, während in der Software-Branche die Nutzer oft mehr Verständnis für komplexere Lizenzierungsprozesse haben.
              \item Vielfalt der Medien: Digitale Inhalte umfassen eine Vielzahl von Medienformen, die jeweils spezifische Schutzmaßnahmen erfordern können. Die Software-Branche konzentriert sich hauptsächlich auf ausführbare Programme.
            \end{itemize}
    \end{enumerate}
  \end{solution}
\end{exercise}

\begin{exercise}{Hardwarebaustein zum Rechtemanagement}
  Mit dem Einsatz eines Hardwarebausteins im PC, der für seinen Besitzer nicht ausforschbar ist, und der aus Inhalteanbieter-Sicht für eine "sichere" Systemkonfiguration sorgt, soll ein digitales Rechtemanagement möglich werden. Beschreiben Sie, wie der Ablauf vom Booten des PCs bis zur Nutzung des Inhalts aussehen könnte, damit keine fremde Software an die ungesicherten Mediendaten kommt.

  \begin{solution}
    \begin{enumerate}
      \item Hardwarebaustein initialisieren: Der Hardwarebaustein wird beim Booten des PCs initialisiert und überprüft die Integrität des Systems.
      \item Signatur überprüfen (Secure Boot): Der Hardwarebaustein überprüft die Signatur des Betriebssystems, um sicherzustellen, dass nur vertrauenswürdige Software gestartet wird.
      \item Betriebssystem starten: Nach erfolgreicher Überprüfung der Signatur wird das Betriebssystem gestartet.
      \item Trusted Execution Environment (TEE) initialisieren: Eine spezielle TEE innerhalb des Betriebssystems sorgt dafür, dass nur autorisierte Anwendungen auf geschützte Inhalte zugreifen können.
      \item Zugriff auf Ressource anfragen: Wenn der Benutzer auf geschützte Mediendaten zugreifen möchte, authentifiziert er sich gegenüber dem System (z.B. über Passwort, Fingerabdruck oder Smartcard).
      \item Benutzer authentifizieren: Die autorisierte Anwendung fordert den Hardwarebaustein auf, die Mediendaten zu entschlüsseln und in einen gesicherten Speicherbereich zu laden, auf den nur diese Anwendung zugreifen kann.
      \item Entschlüsseln durch Hardwarebaustein ODER Blockieren von nicht autorisierter Software: Jegliche Versuche von nicht autorisierter Software, auf diesen Speicherbereich zuzugreifen, werden blockiert.
    \end{enumerate}
  \end{solution}
\end{exercise}

\begin{exercise}{Schranken des Urheberrechts}
  Aus aktuellem Anlass veröffentlichen Sie eine Zusammenstellung aller öffentlichen Reden des Bundespräsidenten und des Innenministers der letzten Wochen in Buchform. Können Sie sich dabei auf die gesetzlich erlaubte Nutzung gemäß § 48 UrhG berufen?

  \begin{solution}
    §48 I 2. UrhG erlaubt die Vervielfältigung, Verbreitung und öffentliche Wiedergabe von Reden, die in öffentlichen Versammlungen gehalten werden. Das ist grds gegeben, allerdings schränkt der §48 II UrhG ein, dass die Reden nicht überwiegend vom selben Verfasser stammen dürfen. Das dürfte hier der Fall sein, da hier eindeutig ein Fokus auf den jeweiligen Verfassern (Bundespräsident \& Innenminister) liegt.

    Daher ist die Veröffentlichung in diesem Fall wohl nicht durch §48 gedeckt.
  \end{solution}
\end{exercise}

\begin{exercise}{Schutzrechte des geistigen Eigentums}
  Wie entsteht ein Schutzrecht an geistigem Eigentum? Vergleichen Sie Urheberrecht und Patentrecht.

  \begin{solution}
    Ein Schutzrecht an geistigem Eigentum entsteht entweder durch die Schaffung eines Werkes (Urheberrecht) oder durch die Anmeldung und Erteilung eines Patents (Patentrecht).
    \begin{itemize}
      \item Urheberrecht: Entsteht automatisch mit der Schaffung eines Werkes, das eine gewisse Schöpfungshöhe erreicht. Es ist nicht erforderlich, das Werk irgendwo anzumelden. Der Schutz umfasst Werke der Literatur, Wissenschaft und Kunst, einschließlich Computerprogrammen, Musik und bildender Kunst.
      \item Patentrecht: Erfordert eine formale Anmeldung und Prüfung durch ein Patentamt. Ein Patent wird für technische Erfindungen erteilt, die neu, erfinderisch und gewerblich anwendbar sind. Der Schutz gewährt dem Inhaber das ausschließliche Recht, die Erfindung zu nutzen und Dritte davon auszuschließen.
    \end{itemize}
  \end{solution}
\end{exercise}

\begin{exercise}{Voraussetzungen für Werkeigenschaft}
  Sie haben ein Computerprogramm entwickelt, das automatisiert kunstvolle Bilder zeichnet. Prüfen Sie für die generierten Bilder und das Computerprogramm, ob die Voraussetzungen für ein urheberrechtlich geschütztes Werk vorliegen.

  \begin{solution}
    \begin{itemize}
      \item Computerprogramm: Programme fallen unter das Urheberrecht, wenn sie eine individuelle geistige Schöpfung darstellen. Ihr Computerprogramm ist wahrscheinlich urheberrechtlich geschützt, da es eine originelle Lösung für die Erstellung kunstvoller Bilder bietet.
      \item Generierte Bilder: Die urheberrechtliche Schutzfähigkeit der generierten Bilder hängt davon ab, ob sie eine ausreichende Schöpfungshöhe erreichen und als individuelle geistige Schöpfungen angesehen werden können. Wenn die Bilder nur durch den Algorithmus ohne wesentlichen menschlichen Beitrag entstehen, könnte der urheberrechtliche Schutz fraglich sein. Wenn jedoch der Programmierer durch die Wahl der Algorithmen und Parameter eine kreative Leistung erbracht hat, könnten die Bilder als Werke geschützt sein.
    \end{itemize}
  \end{solution}
\end{exercise}

\begin{exercise}{Erwerb von Nutzungsrechten}
  \begin{enumerate}
    \item Was versteht man im deutschen Recht allgemein unter gutgläubigem Erwerb?
    \item Wie verhält es sich mit dem gutgläubigen Erwerb im Urheberrecht?
    \item Prüfen Sie, ob folgende Aussage wahr oder falsch ist: "Nutzungsrechte können nur vom Urheber wirksam eingeräumt werden." Begründen Sie!
  \end{enumerate}

  \begin{solution}
    \begin{enumerate}
      \item Gutgläubiger Erwerb: Im deutschen Recht bezeichnet der gutgläubige Erwerb den Erwerb von Eigentum oder anderen Rechten von einem Nichtberechtigten, wenn der Erwerber gutgläubig, d.h. ohne Kenntnis der fehlenden Berechtigung des Veräußerers, handelt und gewisse Voraussetzungen erfüllt sind. Dies ist im Bürgerlichen Gesetzbuch (BGB) in den §§932 ff. geregelt.
      \item Gutgläubiger Erwerb im Urheberrecht: Im Urheberrecht ist der gutgläubige Erwerb von Nutzungsrechten grundsätzlich nicht möglich. Nutzungsrechte können nur vom Rechteinhaber oder von jemandem, der vom Rechteinhaber ermächtigt wurde, übertragen werden. Dies ergibt sich aus §31 UrhG, der die Einräumung von Nutzungsrechten regelt und da es umstritten ist, ob das Abstraktionsprinzip (Verpflichtungs- \& Verfügungsgeschäfte unabhängig) im Urheberrecht gilt. Daher muss die Vertragskette lückenlos nachvollzogen werden.
      \item Diese Aussage ist falsch. Nutzungsrechte können auch von anderen Personen eingeräumt werden, sofern sie vom Urheber die entsprechenden Rechte übertragen bekommen haben. Dies kann durch vertragliche Vereinbarungen oder Lizenzierungen geschehen. Hierzu regelt §31 UrhG die Einräumung und Übertragung von Nutzungsrechten und erlaubt es Rechteinhabern, ihre Rechte weiterzugeben.
    \end{enumerate}
  \end{solution}
\end{exercise}


\begin{exercise}{Erwerb von Nutzungsrechten}
  Ein E-Book-Verlag hat das ausschließliche Nutzungsrecht zur digitalen Vermarktung eines Romans vom Autor erworben. Unter welchen Umständen darf der Verlag einem Dritten ein Nutzungsrecht einräumen?

  \begin{solution}
    Der Verlag darf einem Dritten ein Nutzungsrecht einräumen, wenn dies im Vertrag mit dem Autor ausdrücklich erlaubt wurde (§31 UrhG). Der Autor kann dem Verlag entweder das Recht zur Unterlizenzierung einräumen oder dies ausschließen. Ohne eine ausdrückliche Erlaubnis zur Unterlizenzierung im Vertrag ist der Verlag nicht berechtigt, einem Dritten Nutzungsrechte einzuräumen (§35 UrhG).
  \end{solution}
\end{exercise}

\begin{exercise}{Lizenzen}
  \begin{enumerate}
    \item Nennen Sie zwei Vorteile der Standardisierung von Nutzungslizenzen wie Creative Commons und erläutern Sie diese kurz.
    \item Wieso gibt es die Variation der Creative-Commons-Lizenz CC-BY-ND-SA nicht?
    \item Darf ein gemeinfreies Werk unter der Lizenz CC0 auch in einem Werk nach CC-BY-SA verwendet werden? Begründen Sie.
  \end{enumerate}

  \begin{solution}
    \begin{enumerate}
      \item Vorteile der Standardisierung von Nutzungslizenzen:
            \begin{itemize}
              \item Rechtssicherheit: Standardisierte Lizenzen bieten klare und verständliche Bedingungen, die sowohl für Urheber als auch für Nutzer transparent sind, wodurch rechtliche Unsicherheiten minimiert werden.
              \item Einfache Handhabung: Durch die Verwendung standardisierter Lizenzen wie Creative Commons müssen Urheber keine eigenen, komplexen Lizenztexte erstellen. Nutzer können schnell erkennen, welche Rechte und Pflichten sie haben.
            \end{itemize}
      \item Variation CC-BY-ND-SA: Diese Variation existiert nicht, weil sie widersprüchlich ist. "ND" (No Derivatives) bedeutet, dass keine Bearbeitungen des Werkes erlaubt sind, während "SA" (Share Alike) vorschreibt, dass bearbeitete Versionen unter denselben Bedingungen lizenziert werden müssen. Diese beiden Bedingungen schließen sich gegenseitig aus.
      \item Gemeinfreies Werk unter CC0 und CC-BY-SA: Ja, ein gemeinfreies Werk, das unter der Lizenz CC0 steht, kann in einem Werk nach CC-BY-SA verwendet werden. CC0 verzichtet auf alle Urheberrechte, sodass das Werk frei verwendet werden kann, auch in Werken, die unter restriktiveren Lizenzen stehen.
    \end{enumerate}
  \end{solution}
\end{exercise}

\begin{exercise}{Nutzung von Computerprogrammen}
  Ein Lizenznehmer möchte ein Computerprogramm zur automatisierten Prüfung auf Sicherheitslücken dekompilieren. Unter welchen Voraussetzungen wäre eine solche Dekompilierung urheberrechtlich zulässig?

  \begin{solution}
    Die Dekompilierung eines Computerprogramms ist nach §69e UrhG unter bestimmten Voraussetzungen zulässig:
    \begin{itemize}
      \item Berechtigte Person (I 1.): Die Dekompilierung darf nur von einer berechtigten Person durchgeführt werden, d.h. vom rechtmäßigen Lizenznehmer des Programms.
      \item Notwendige Informationen nicht verfügbar (I 2.): Die Dekompilierung ist nur zulässig, wenn die für die Herstellung der Interoperabilität notwendigen Informationen nicht auf andere Weise zugänglich sind.
      \item Beschränkung auf bestimmte Teile (I 3.): Die Dekompilierung darf sich nur auf die Teile des Programms beschränken, die für die Herstellung der Interoperabilität erforderlich sind.
      \item Keine anderen Zwecke (II 1.): Die gewonnenen Informationen dürfen nur für die Herstellung der Interoperabilität und nicht für andere Zwecke verwendet werden.
      \item Keine Weitergabe an Dritte (II 2.): Die gewonnenen Informationen dürfen nicht an Dritte weitergegeben werden, es sei denn, dies ist zur Herstellung der Interoperabilität erforderlich.
      \item Entwicklung unabhängiger Programme (II 3.): Die gewonnenen Informationen dürfen nur zur Entwicklung unabhängiger Programme verwendet werden und nicht zur Schaffung eines abgeleiteten Werks.
    \end{itemize}

    Diese Voraussetzungen scheinen nicht gegeben zu sein, da die Dekompilierung zur automatisierten Prüfung auf Sicherheitslücken nicht unbedingt der Herstellung der Interoperabilität dient.
  \end{solution}
\end{exercise}

\begin{exercise}{Werke}
  Bei einem Live-Auftritt rappt ein Künstler den Text einer öffentlichen Bundestagsrede Angela Merkels zu seiner elektronischen Bearbeitung von Giuseppe Verdis Werk Dies Irae.
  \begin{enumerate}
    \item Nennen Sie vier Werke, die bei dieser Ausführung verwertet oder geschaffen werden.
    \item Hat der Künstler bei der Vorführung Urheberrechte Dritter verletzt? Begründen Sie.
    \item Ein Zuschauer macht während der Show Fotos des Künstlers und veröffentlicht diese im Internet. Darf er das? Begründen Sie.
  \end{enumerate}

  \begin{solution}
    \begin{enumerate}
      \item Vier Werke, die bei dieser Ausführung verwertet oder geschaffen werden:
            \begin{itemize}
              \item Der Originaltext der Bundestagsrede von Angela Merkel.
              \item Das musikalische Werk "Dies Irae" von Giuseppe Verdi.
              \item Die elektronische Bearbeitung von "Dies Irae".
              \item Das neu geschaffene Rap-Stück, das Text und Musik kombiniert.
            \end{itemize}
      \item Ja, der Künstler hat bei der Vorführung Urheberrechte Dritter verletzt. Die Nutzung des Textes der Bundestagsrede könnte eine Urheberrechtsverletzung darstellen, wenn die Rede urheberrechtlich geschützt ist. Ebenso bedarf die Nutzung und Bearbeitung von Giuseppe Verdis "Dies Irae" der Zustimmung der Rechteinhaber, sofern das Werk nicht gemeinfrei ist.
      \item Ein Zuschauer darf grundsätzlich Fotos des Künstlers machen und veröffentlichen, wenn der Auftritt in einem öffentlichen Raum stattfindet und keine vertraglichen Vereinbarungen oder Hausregeln dagegen sprechen. Allerdings muss das Persönlichkeitsrecht des Künstlers berücksichtigt werden, und kommerzielle Nutzung der Fotos bedarf der Zustimmung des Künstlers.
    \end{enumerate}
  \end{solution}
\end{exercise}

\begin{exercise}{Serienstreaming}
  \begin{enumerate}
    \item Der Nutzer Serienfan300 schaut gerne Serien der Streamingplattform Webfluxilux auf seinem Tablet. Findet dabei eine Vervielfältigung im Sinne des Urheberrechtes statt? Diskutieren Sie mögliche Fallunterscheidungen und beurteilen Sie deren Zulässigkeit. Begründen Sie.
    \item Dieser Serienfan300 betreibt einen öffentlichen Blog, auf dem er wöchentlich selbst angefertigte Screenshots (Einzelbilder) seiner liebsten Serien postet, die er beim Nutzen von Webfluxilux erstellt. Begeht er hierbei eine Urheberrechtsverletzung? Begründen Sie.
    \item Wie bewerten Sie die Rechtmäßigkeit, wenn Serienfan300 in seinem Blog anstelle selbst angefertigter Screenshots auf öffentlich verfügbare Screenshots und Videos anderer Nutzer verweist (verlinkt), ohne diese in seinem Blog direkt sichtbar zu machen?
    \item Wie bewerten Sie die Rechtmäßigkeit, wenn der Serienfan300 in seinem Blog das Kostümdesign in den Serien bespricht und zur Veranschaulichung Videoclips einbettet?
    \item Auf TheyTube gibt es den Benutzer AllergroßterSerienfan, der mehrminütige Videoausschnitte aus Webflux-eigenen Serien hochlädt. Serienfan300 wird verdächtigt, dieser Benutzer zu sein. Gibt es eine technische Möglichkeit für Webflux, zu ermitteln, welcher Benutzer Videoclips unrechtmäßig veröffentlicht hat? Erläutern Sie die Funktion.
  \end{enumerate}

  \begin{solution}
    \begin{enumerate}
      \item Wenn Serienfan300 Serien auf Webfluxilux streamt, findet technisch gesehen eine temporäre Vervielfältigung auf seinem Tablet statt, da Daten in den Zwischenspeicher (Cache) geladen werden. Nach §44a UrhG sind solche vorübergehenden Vervielfältigungen jedoch zulässig, wenn sie flüchtig oder begleitend sind und einen integralen und wesentlichen Teil eines technischen Verfahrens darstellen. Somit ist das Streaming selbst zulässig und stellt keine Urheberrechtsverletzung dar.
      \item Ja, Serienfan300 begeht eine Urheberrechtsverletzung, wenn er selbst angefertigte Screenshots von geschützten Serien auf seinem öffentlichen Blog postet. Diese Screenshots stellen Vervielfältigungen nach §16 UrhG dar und das öffentliche Zugänglichmachen nach §19a UrhG. Ohne Erlaubnis des Rechteinhabers sind solche Handlungen nicht zulässig.
      \item Wenn Serienfan300 auf öffentlich verfügbare Screenshots und Videos anderer Nutzer verlinkt, ohne diese direkt auf seinem Blog sichtbar zu machen, ist dies rechtlich unproblematisch, solange die verlinkten Inhalte rechtmäßig veröffentlicht wurden. Nach §44a UrhG und den Grundsätzen des Hyperlinkings kann das Verlinken auf rechtmäßige Inhalte erlaubt sein.
      \item Serienfan300 darf Videoclips nicht einbetten, um das Kostümdesign zu besprechen, da dies eine Vervielfältigung und ein öffentliches Zugänglichmachen nach §§16, 19a UrhG darstellt und ohne Erlaubnis des Rechteinhabers unzulässig ist. Es gibt allerdings Ausnahmen, wie das Zitatrecht nach §51 UrhG, die aber hier wahrscheinlich nicht greifen, da das Einbetten ganzer Clips über die zulässige Zitierung hinausgeht.
      \item Webflux kann technische Maßnahmen einsetzen, um zu ermitteln, welcher Benutzer Videoclips unrechtmäßig veröffentlicht hat. Eine Möglichkeit ist das Watermarking, bei dem unsichtbare Wasserzeichen in die Videodaten eingebettet werden. Diese Wasserzeichen können Informationen über den ursprünglichen Benutzer enthalten, der das Video gestreamt hat, sodass Webflux den Ursprung der illegalen Veröffentlichung nachverfolgen kann.
    \end{enumerate}
  \end{solution}
\end{exercise}

\begin{exercise}{Barrierefreiheit}
  Herr Mac Mole kauft sich einen neuen Bestseller-Roman der Schriftstellerin Conni Funkel. Da er selbst mit Brille sehr schlecht sieht, lässt er sich das Buch ohne Erlaubnis der Autorin von Montgomery Arron, einem Freund, als Hörbuch aufnehmen.
  \begin{enumerate}
    \item Hat einer der beiden dabei gegen das Urheberrecht verstoßen? Begründen Sie.
    \item Einige ebenfalls schlechtsichtige Freunde Mac Moles sind von der Idee begeistert und bitten ihn, das entstandene Hörbuch an sie weiterzugeben. Ist Mac Mole dazu berechtigt? Begründen Sie.
    \item Die Buchautorin erfährt von dem Hörbuch und fordert von Mac Mole eine Vergütung für die vervielfältigten Werke. Hat sie nach den §§45a bis 45d UrhG einen Anspruch auf Vergütung gegenüber Mac Mole? Begründen Sie.
    \item Darf Mac Mole das Hörbuch öffentlich anbieten und dafür Geld verlangen? Begründen Sie.
  \end{enumerate}

  \begin{solution}
    \begin{enumerate}
      \item Nein, keiner der beiden hat gegen das Urheberrecht verstoßen. Nach §45a UrhG dürfen Werke für Personen mit Behinderungen in einer für sie zugänglichen Form vervielfältigt und verbreitet werden, wenn dies unmittelbar mit der Behinderung zusammenhängt, nicht kommerziellen Zwecken dient und in dem erforderlichen Umfang geschieht. Da Mac Mole das Buch aufgrund seiner Sehbehinderung als Hörbuch benötigt und dies nicht kommerziellen Zwecken dient, ist dies erlaubt. Dies ist insbesondere im §45b geregelt.
      \item Nein, Mac Mole ist nicht berechtigt, das entstandene Hörbuch an seine ebenfalls sehbehinderten Freunde weiterzugeben. Die Ausnahmeregelung des §45a UrhG gilt nur für die Herstellung und Nutzung durch die betroffene Person selbst. Eine Weitergabe an Dritte ist nicht gestattet.
      \item Nein, die Buchautorin hat keinen Anspruch auf Vergütung. Nach §45a II UrhG ist der Urheber nicht zur Zahlung einer Vergütung berechtigt, wenn die Vervielfältigung im Rahmen der gesetzlichen Schrankenregelung aus §45b-d UrhG für behinderte Personen erfolgt.
      \item Nein, Mac Mole darf das Hörbuch nicht öffentlich anbieten und dafür Geld verlangen. Nach §45a UrhG ist die Nutzung nur für nicht kommerzielle Zwecke gestattet. Das öffentliche Anbieten und der Verkauf des Hörbuchs würde einen kommerziellen Zweck darstellen und wäre daher nicht erlaubt.
    \end{enumerate}
  \end{solution}
\end{exercise}


\sheet[2023]{Altklausur}
\begin{exercise}{Schutz an geistigem Eigentum}
  \begin{enumerate}
    \item Warum gibt es "geistiges Eigentum"? Wie lässt sich der Schutz von diesem begründen?
    \item Wie entsteht ein Schutzrecht? Vergleichen Sie Urheberrecht und Patentrecht.
  \end{enumerate}

  \begin{solution}
    \begin{enumerate}
      \item Geistiges Eigentum existiert, um die immateriellen Ergebnisse kreativer und innovativer Aktivitäten zu schützen. Der Schutz lässt sich durch mehrere Gründe begründen:
            \begin{itemize}
              \item Anreiz für Kreativität und Innovation: Der Schutz geistigen Eigentums schafft Anreize für Erfinder und Künstler, indem er ihnen Exklusivrechte gewährt, die wirtschaftliche Erträge ermöglichen.
              \item Investitionsschutz: Geistiges Eigentum schützt Investitionen in Forschung und Entwicklung, indem es sicherstellt, dass die Investoren von ihren Ausgaben profitieren können.
              \item Wirtschaftliche Entwicklung: Der Schutz fördert den technischen Fortschritt und die wirtschaftliche Entwicklung durch die Förderung von Innovationen und die Schaffung neuer Märkte.
              \item Rechtliche Anerkennung: Der Schutz bietet rechtliche Anerkennung und Unterstützung für die Rechte von Schöpfern und Erfindern.
            \end{itemize}
      \item Ein Schutzrecht entsteht durch verschiedene Verfahren, die je nach Art des geistigen Eigentums variieren:
            \begin{itemize}
              \item Urheberrecht:
                    \begin{itemize}
                      \item Entstehung: Automatisch mit der Schaffung eines Werkes, wenn es eine persönliche geistige Schöpfung darstellt.
                      \item Dauer: In der Regel bis 70 Jahre nach dem Tod des Urhebers.
                      \item Schutzgegenstand: Originale literarische, wissenschaftliche und künstlerische Werke.
                    \end{itemize}
              \item Patentrecht:
                    \begin{itemize}
                      \item Entstehung: Durch formale Anmeldung und Prüfung beim Patentamt.
                      \item Dauer: In der Regel 20 Jahre ab Anmeldedatum.
                      \item Schutzgegenstand: Technische Erfindungen, die neu, erfinderisch und gewerblich anwendbar sind.
                    \end{itemize}
            \end{itemize}
    \end{enumerate}
  \end{solution}
\end{exercise}

\begin{exercise}{Schranken des Urheberrechts}
  Eine Mediathek möchte (alte) Filme digitalisieren und online veröffentlichen. Die Urheber der Filme sind nicht ausfindig zu machen. Die Filme gelten nach §61ff. UrhG als verwaiste Werke.
  \begin{enumerate}
    \item Darf die Mediathek die Filme kostenlos zur Verfügung stellen?
    \item Die Mediathek möchte die Filme unter CC-BY lizensieren (Creative-Commons, Nennung des Urhebers). Da die Urheber nicht bekannt sind, soll hier die Mediathek als Urheber angegeben werden. Ist diese Lizensierung zulässig?
    \item Darf die Mediathek für die Nutzung der Filme ein Entgelt von den Nutzern verlangen?
    \item Muss eine Vergütung an eine Verwertungsgesellschaft gezahlt werden?
    \item Eine Filmhochschule möchte eine Datenbank mit Metainformationen und historischen Kontexten zu den Filmen in der Mediathek erstellen. Diese Datenbank soll auch Links zu den Filmen enthalten.
          \begin{enumerate}
            \item Muss sich die Filmhochschule eine Erlaubnis der Mediathek einholen?
            \item Darf die Filmhochschule ein Entgelt für die Nutzung der Datenbank verlangen?
          \end{enumerate}
  \end{enumerate}

  \begin{solution}
    \begin{enumerate}
      \item Ja, die Mediathek darf verwaiste Werke digitalisieren und online zur Verfügung stellen, allerdings sind dabei bestimmte Bedingungen zu beachten. Nach §61b UrhG dürfen Einrichtungen wie Mediatheken verwaiste Werke nutzen, wenn sie nach einer sorgfältigen und dokumentierten Suche die Urheber nicht ausfindig machen konnten. Dies bedeutet, dass die Mediathek die Filme kostenlos zur Verfügung stellen darf, sofern diese Voraussetzungen erfüllt sind.
      \item Nein, diese Lizenzierung ist nicht zulässig. Die Mediathek kann sich nicht als Urheber der Werke ausgeben, da sie nicht die tatsächlichen Urheber der Filme ist. Die Nutzung von verwaisten Werken erlaubt es nicht, neue urheberrechtliche Ansprüche zu erheben oder die Werke unter einer Lizenz zu veröffentlichen, die eine falsche Urheberangabe enthält. Somit darf die Mediathek die Filme nicht unter CC-BY lizensieren und sich selbst als Urheber angeben.
      \item Ja, die Mediathek darf für die Nutzung der digitalisierten Filme ein Entgelt verlangen. §61d II UrhG erlaubt es, dass Einrichtungen, die verwaiste Werke nutzen, dies auch gegen ein angemessenes Entgelt tun können. Die Einnahmen sollten jedoch zur Deckung der Kosten der Digitalisierung und der Bereitstellung verwendet werden.
      \item Ja, eine Vergütung an eine Verwertungsgesellschaft ist zu zahlen. Nach §61f UrhG haben die Urheber oder deren Rechtsnachfolger, wenn sie später identifiziert werden, Anspruch auf eine angemessene Vergütung für die Nutzung des Werkes. Diese Vergütung wird in der Regel durch Verwertungsgesellschaften organisiert und verwaltet. Die Mediathek muss also sicherstellen, dass entsprechende Rückstellungen oder Zahlungen an die Verwertungsgesellschaft erfolgen, falls die Urheber später ermittelt werden.
      \item
            \begin{enumerate}
              \item Nein, eine Erlaubnis der Mediathek ist nicht erforderlich, wenn die Filmhochschule in ihrer Datenbank lediglich Links zu den Filmen in der Mediathek bereitstellt und keine Kopien oder anderweitige Nutzungen der Filme vornimmt. Das Verlinken auf rechtmäßig zugänglich gemachte Inhalte stellt in der Regel keine urheberrechtlich relevante Nutzung dar, die einer Genehmigung bedarf.
              \item Ja, die Filmhochschule darf ein Entgelt für die Nutzung der Datenbank verlangen. Solange die Nutzung der Datenbank (einschließlich der enthaltenen Links zu den Filmen) im Einklang mit den Rechten und Vereinbarungen steht, die die Mediathek gewährt hat, steht es der Filmhochschule frei, für die Bereitstellung und Nutzung ihrer Datenbank eine Gebühr zu erheben.
            \end{enumerate}
    \end{enumerate}
  \end{solution}
\end{exercise}

\begin{exercise}{Digitale Wasserzeichen}
  Eine Online-Galerie will eine limitierte Auflage digitaler Kopien eines Werkes einer Künstlerin verkaufen.
  \begin{enumerate}
    \item Bewerten sie das potentielle Schutzniveau der folgenden Vorgehensweisen:
          \begin{enumerate}
            \item In das Bild wird ein digitales Watermark eingebettet, beim Kaufen wird ein Zähler hochgezählt, damit maximal 100 Exemplare verkauft werden. Alle Exemplare enthalten dasselbe Watermark.
            \item Im Gegensatz zu 1 wird die Nummer des Exemplars im letzten Byte der Datei gespeichert.
            \item In jedem Exemplar wird ein individuelles Watermark eingebettet. Alle Käufer*innen erhalten ein gleichwertiges, aber individuelles Werk.
          \end{enumerate}
    \item Beschreiben Sie, wie ein Prozess zur automatisierten Aufspürung illegaler Kopien bei diesen Vorgehensweisen jeweils aussehen könnte.
  \end{enumerate}

  \begin{solution}
    \begin{enumerate}
      \item Schutzniveau der Vorgehensweisen:
            \begin{enumerate}
              \item Gleiches Watermark für alle Exemplare: Das Schutzniveau ist gering, da alle Kopien dasselbe Watermark enthalten. Dies erschwert die Rückverfolgbarkeit auf einzelne Käufer bei einer unautorisierten Verbreitung.
              \item Nummer des Exemplars im letzten Byte: Das Schutzniveau ist leicht erhöht, da jedes Exemplar eine eindeutige Kennung besitzt. Dies kann jedoch leicht manipuliert werden, da die Position des Watermarks bekannt ist.
              \item Individuelles Watermark für jedes Exemplar: Das Schutzniveau ist am höchsten, da jede Kopie ein einzigartiges Watermark enthält. Dies ermöglicht eine genaue Rückverfolgbarkeit und erschwert die Manipulation.
            \end{enumerate}
      \item Prozess zur automatisierten Aufspürung illegaler Kopien:
            \begin{enumerate}
              \item Gleiches Watermark für alle Exemplare: Eine automatisierte Suche nach dem Watermark in Online-Medien kann alle Kopien identifizieren. Allerdings ist es nicht möglich, den ursprünglichen Käufer zu ermitteln.
              \item Nummer des Exemplars im letzten Byte: Eine Software könnte alle Bytes am Ende von Bilddateien überprüfen, um die eindeutige Nummer zu finden. Dies ermöglicht die Identifizierung des ursprünglichen Käufers, sofern die Nummer nicht manipuliert wurde.
              \item Individuelles Watermark für jedes Exemplar: Eine Software kann spezifische Watermarks in den Bildern suchen und diese mit einer Datenbank abgleichen, die die Watermarks den Käufern zuordnet. Dies erlaubt eine genaue Rückverfolgung bis zum ursprünglichen Käufer.
            \end{enumerate}
    \end{enumerate}
  \end{solution}
\end{exercise}



\sheet[2020]{Altklausur}
\begin{exercise}{Werke und Urheberschaft}
  Forscher:innen haben eine KI mit Werken des Malers Vicent van Gogh darauf trainiert, neue Werke in dessen Stil zu kreieren. Sie veröffentlichen die Bilder unter dem Namen "van Gogh posthum".
  \begin{enumerate}
    \item Gelten diese neu produzierten Bilder als geschützte Werke nach dem Urheberrecht?
    \item Ist die KI selbst ein solch geschütztes Werk? Begründen Sie.
  \end{enumerate}

  \begin{solution}
    \begin{enumerate}
      \item Die neu produzierten Bilder gelten nicht als geschützte Werke nach dem Urheberrecht im traditionellen Sinne, da sie von einer KI und nicht von einem menschlichen Schöpfer erstellt wurden. Nach deutschem Urheberrecht setzt der Schutz eines Werkes eine persönliche geistige Schöpfung voraus, was bedeutet, dass der Urheber ein Mensch sein muss. Allerdings könnten die Bilder durch verwandte Schutzrechte oder das Datenbankrecht geschützt sein.
      \item Die KI selbst kann als Computerprogramm urheberrechtlich geschützt sein, vorausgesetzt, sie erfüllt die Kriterien einer persönlichen geistigen Schöpfung. Dies bedeutet, dass die Entwickler der KI die Urheberrechte an der Software haben, da sie die schöpferische Leistung erbracht haben. Der Schutzumfang der KI umfasst dann den Quellcode und die Programmstruktur.
    \end{enumerate}
  \end{solution}
\end{exercise}

\begin{exercise}{Schranken des Urheberrechts}
  Ein Foto-Künstler ist beeindruckt von Graffiti, das auf öffentlich einsehbare Hausfassaden gesprüht ist.
  \begin{enumerate}
    \item Durfte er die Graffiti-Fotos ohne Lizenz veröffentlichen? Begründen Sie.
    \item Herr U lässt eine Kopie eines der Graffiti auf seine Fassade sprühen. Wie ändert sich die rechtliche Lage?
    \item Eine Graffiti-Künstlerin erkennt ihr Werk in dem Bildband. Sie beschwert sich, dass ihr Künstler-Name hätte genannt werden müssen. Unter welchen Bedingungen hätte ihr Name genannt werden müssen?
    \item Für einen zweiten Bildband will der Fotograf mit Hilfe einer Drohne Graffitis von unterschiedlichen Blickwinkeln aus der Luft fotografieren. Was muss er dabei beachten, um keinen Urheberrechts-Verstoß zu begehen?
  \end{enumerate}

  \begin{solution}
    \begin{enumerate}
      \item Ja, der Foto-Künstler durfte die Graffiti-Fotos ohne Lizenz veröffentlichen, sofern die Graffitis sich dauerhaft an öffentlichen Orten befinden. Gemäß §59 UrhG (Panoramafreiheit) dürfen Werke, die sich bleibend an öffentlichen Wegen, Straßen oder Plätzen befinden, fotografiert und die Fotos ohne Zustimmung des Urhebers veröffentlicht werden.
      \item Wenn Herr U eine Kopie eines der Graffiti auf seine Fassade sprühen lässt, handelt es sich um eine Vervielfältigung des Werkes. Dies erfordert die Zustimmung des Urhebers, da es sich nicht mehr um eine zulässige Nutzung im Rahmen der Panoramafreiheit handelt. Somit liegt hier ein Urheberrechtsverstoß vor, sofern keine Erlaubnis eingeholt wurde.
      \item Der Name der Graffiti-Künstlerin hätte genannt werden müssen, wenn sie dies nach §13 UrhG ausdrücklich verlangt hat und es für die Art und Weise der Nutzung üblich ist. Da dies bei künstlerischen Werken oft der Fall ist, wäre die Nennung ihres Namens erforderlich gewesen, um ihr Recht auf Anerkennung der Urheberschaft zu wahren.
      \item Beim Fotografieren der Graffitis aus der Luft mit einer Drohne muss der Fotograf sicherstellen, dass er keine urheberrechtlich geschützten Werke in einer Weise abbildet, die über die Panoramafreiheit hinausgeht. Dies bedeutet, dass die Graffitis bleibend öffentlich zugänglich sein müssen und die Nutzung die Rechte des Urhebers nicht unzumutbar beeinträchtigen darf. Zudem sind datenschutzrechtliche Bestimmungen und gegebenenfalls Flugverbotszonen zu beachten.
    \end{enumerate}
  \end{solution}
\end{exercise}

\begin{exercise}{eLearning}
  Professorin P nimmt es mit Urheberrecht bei der Bebilderung ihrer Vorlesungsfolien nicht so genau. Daher befinden sich in den Folien urheberrechtlich geschützte Bildwerke Dritter, für die keine Lizenz vorliegt.
  \begin{enumerate}
    \item Nun lädt P ihre Folien auf eine eLearning-Plattform hoch, auf die nur ihre Studierenden Zugriff haben. Ist das rechtens? Begründen Sie.
    \item Angenommen die Folien wären öffentlich abrufbar. Wie verändert sich die urheberrechtliche Situation?
  \end{enumerate}

  \begin{solution}
    \begin{enumerate}
      \item Das Hochladen der Folien auf eine eLearning-Plattform, die nur für ihre Studierenden zugänglich ist, könnte gemäß §60a UrhG erlaubt sein. Diese Regelung ermöglicht die Nutzung urheberrechtlich geschützter Werke zu Zwecken der Bildung und Wissenschaft unter bestimmten Bedingungen, wie dem begrenzten Zugriff und der nicht-kommerziellen Nutzung. Es ist jedoch ratsam, die erlaubten Umfang und Nutzung genau zu prüfen.
      \item Wenn die Folien öffentlich abrufbar wären, würde dies eine Verletzung des Urheberrechts darstellen, da die geschützten Bildwerke ohne Erlaubnis der Urheber oder Lizenzgeber genutzt werden. Die Ausnahme des §60a UrhG gilt nicht für die öffentliche Zugänglichmachung, was eine ausdrückliche Zustimmung der Rechteinhaber erforderlich macht.
    \end{enumerate}
  \end{solution}
\end{exercise}



\end{document}