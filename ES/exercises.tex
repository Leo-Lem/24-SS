\documentclass{article}

\usepackage[solutions]{xrcise}

\subject{Eingebettete Systeme}
\semester{Summer 2024}
\author{Leopold Lemmermann}

\begin{document}\createtitle

\sheet{Mafiasi questions}
\begin{exercise}{Embedded Systems Design}
  \begin{enumerate}
    \item Worauf sollte man beim Entwurf von eingebetteten Systemen achten und was sind die Besonderheiten im Vergleich zur normalen Softwareentwicklung?
    \item Wie unterscheiden sich eingebettete Systeme von normalen Computern?
    \item Was versteht man unter Zuverlässigkeit von Embedded Systems und wie kann man diese messen?
    \item Was beschreibt der Fault in Time (FIT) Wert und wie wird dieser bestimmt?
    \item Was versteht man genau unter Scratch Pad Memory?
  \end{enumerate}

  \begin{solution}
    \begin{enumerate}
      \item Besonderheiten: Echtzeitfähigkeit, Ressourcenknappheit, Energieeffizienz, Sicherheit
      \item Unterschiede: Eingebettete Systeme sind spezialisierte Computersysteme, die in anderen Systemen eingebettet sind und spezielle Aufgaben erfüllen.
      \item Zuverlässigkeit: Zuverlässigkeit beschreibt die Fähigkeit eines Systems, eine bestimmte Funktion unter spezifizierten Bedingungen
      \item FIT: Der FIT-Wert gibt die Anzahl der Ausfälle pro Milliarde Betriebsstunden an und wird durch Zuverlässigkeitstests bestimmt.
      \item Scratch Pad Memory: Ein Scratchpad-Speicher ist ein kleiner, schneller und energieeffizienter Speicher, der zur Speicherung häufig verwendeter Daten verwendet wird.
    \end{enumerate}
  \end{solution}
\end{exercise}

\begin{exercise}{Betriebssysteme}
  \begin{enumerate}
    \item Worin Unterscheiden sich Betriebssysteme von Eingebetteten Systemen und "normalen" Betriebssystemen?
    \item Wie behandelt ein normales Betriebssystem Interrupts?
    \item Was machen Betriebsysteme noch?
    \item Wie unterscheiden sich hier die normalen Betriebssysteme von den eingebetteten Systemen und wo liegen die Vorteile der einzelnen Systeme?
  \end{enumerate}

  \begin{solution}
    \begin{enumerate}
      \item Unterschiede: Eingebettete Betriebssysteme sind speziell für die Anforderungen von eingebetteten Systemen optimiert und bieten Funktionen wie Echtzeitfähigkeit, geringen Speicherbedarf und Energieeffizienz.
      \item Interrupts: Interrupts werden von einem normalen Betriebssystem behandelt, indem sie den aktuellen Prozess unterbrechen und den Interrupt-Handler ausführen.
      \item Betriebssysteme verwalten Ressourcen wie Prozessoren, Speicher und Geräte, bieten Schnittstellen für Anwendungen und sorgen für die Sicherheit und Zuverlässigkeit des Systems.
      \item Unterschiede: Eingebettete Betriebssysteme sind speziell für die Anforderungen von eingebetteten Systemen optimiert und bieten Funktionen wie Echtzeitfähigkeit, geringen Speicherbedarf und Energieeffizienz.
    \end{enumerate}
  \end{solution}
\end{exercise}

\begin{exercise}{MoCs}
  \begin{enumerate}
    \item Was sind die Beschriftungen der Tabelle über MoCs?
    \item Erklären Sie ein MoC Ihrer Wahl genauer.
    \item Beschreiben Sie kurz, was Shared Memory und Message Passing sind.
    \item Was sind die Vor- und Nachteile von Shared Memory bzw. Message Passing?
    \item Wie bestimmt man die Größe der Buffer beim asynchronen Message Passing?
    \item Was ist Priority Inversion und wie kann man das vermeiden?
    \item Was kann noch für ein Problem auftreten und wie vermeidet man das?
  \end{enumerate}

  \begin{solution}
    \begin{enumerate}
      \item Beschriftungen: Name, Beschreibung, Eigenschaften, Anwendungen, Vor- und Nachteile
      \item MoC: StateCharts
      \item Shared Memory: Shared Memory ist eine Kommunikationstechnik, bei der mehrere Prozesse auf gemeinsame Speicherbereiche zugreifen können. Message Passing: Message Passing ist eine Kommunikationstechnik, bei der Prozesse Nachrichten austauschen, um Daten zu übertragen.
      \item Vor- und Nachteile: Shared Memory ermöglicht schnellen Datenaustausch, kann aber zu
      \item Größe der Buffer: Die Größe der Buffer beim asynchronen Message Passing wird durch die Anzahl der Nachrichten und die Größe der Nachrichten bestimmt.
      \item Priority Inversion: Priority Inversion tritt auf, wenn ein niedrig priorisierter Prozess eine Ressource blockiert, die von einem höher priorisierten Prozess benötigt wird. Priority Inversion kann durch die Verwendung von Prioritätsvererbung oder Prioritätsdecke vermieden werden.
      \item Deadlock: Deadlocks können auftreten, wenn Prozesse auf gegenseitige Exklusivität warten. Deadlocks können durch die Verwendung von Ressourcenverwaltungsalgorithmen wie Banker's Algorithm oder durch die Vermeidung von zyklischen Wartebedingungen vermieden werden.
    \end{enumerate}
  \end{solution}
\end{exercise}

\begin{exercise}{Scheduling}
  \begin{enumerate}
    \item Was sind die 4 Eigenschaften von Scheduling-Verfahren? Beschreiben Sie diese kurz.
    \item Stellen Sie einen Algorithmus vor und ordnen Sie ihn in die Kategorien ein.
  \end{enumerate}

  \begin{solution}
    \begin{enumerate}
      \item Eigenschaften: Determinismus, Fairness, Echtzeitfähigkeit, Effizienz
      \item Rate Monotonic Scheduling: Rate Monotonic Scheduling ist ein deterministisches, präemptives Scheduling-Verfahren, bei dem die Priorität eines Prozesses durch seine Rate bestimmt wird.
    \end{enumerate}
  \end{solution}
\end{exercise}

\begin{exercise}{High-Level Optimierung}
  \begin{enumerate}
    \item Nennen Sie einige High-Level Optimierungen und erklären Sie diese kurz.
    \item Weswegen würde man Loop Unrolling anwenden?
  \end{enumerate}

  \begin{solution}
    \begin{enumerate}
      \item High-Level Optimierungen: Loop Unrolling, Loop Tiling, Data Prefetching
      \item Loop Unrolling: Loop Unrolling wird angewendet, um den Overhead von Schleifen zu reduzieren und die Parallelität von Instruktionen zu erhöhen.
    \end{enumerate}
  \end{solution}
\end{exercise}

\sheet{Einführung (Marwedel)}
\begin{exercise}{Definitionen}
  Definieren Sie die Begriffe "embedded system", "cyber-physical system (CPS)", "Internet of Things (IoT)", und "Industry 4.0".

  \begin{solution}
    Ein eingebettetes System ist ein Computersystem, das in einem größeren System eingebettet ist und spezielle Aufgaben erfüllt.

    Ein Cyber-Physical System (CPS) ist ein System, das die physische Welt mit der virtuellen Welt verbindet.

    Das Internet der Dinge (IoT) ist ein Netzwerk von miteinander verbundenen Geräten, die Daten austauschen und miteinander kommunizieren.

    Industrie 4.0 ist ein Konzept, das die Integration von Informationstechnologie und industrieller Produktion beschreibt.
  \end{solution}
\end{exercise}

\begin{exercise}{Relevanz}
  Warum benötigt man eingebettete Systeme?

  \begin{solution}
    Eingebettete Systeme werden benötigt, um spezielle Aufgaben in anderen Systemen zu erfüllen, wie z.B. in der Automobilindustrie, der Medizintechnik, und der Industrie.
  \end{solution}
\end{exercise}

\begin{exercise}{Herausforderungen}
  Welche Herausforderungen müssen überwunden werden, um die Chancen von eingebetteten Systemen voll auszuschöpfen?

  \begin{solution}
    Herausforderungen, die überwunden werden müssen, um die Chancen von eingebetteten Systemen voll auszuschöpfen, sind z.B. Echtzeitfähigkeit, Energieeffizienz, und Sicherheit.
  \end{solution}
\end{exercise}

\begin{exercise}{Timing Constraints}
  Was sind harte und weiche Timing-Constraints?

  \begin{solution}
    Harte Timing-Constraints sind Timing-Anforderungen, die unbedingt eingehalten werden müssen, während weiche Timing-Constraints weniger streng sind und toleriert werden können.
  \end{solution}
\end{exercise}

\begin{exercise}{Adaptives Sampling}
  Was ist adaptives Sampling?

  \begin{solution}
    Adaptives Sampling ist eine Technik, bei der die Abtastrate eines Sensors an die Änderungen der Umgebung angepasst wird.
  \end{solution}
\end{exercise}

\begin{exercise}{Unterschiede zwischen PC-basierten und eingebetteten Anwendungen}
  Was sind die Hauptunterschiede zwischen PC-basierten Anwendungen und eingebetteten/CPS-Anwendungen?

  \begin{solution}
    Die Hauptunterschiede zwischen PC-basierten Anwendungen und eingebetteten/CPS-Anwendungen sind die Echtzeitfähigkeit, die Ressourcenknappheit, und die Energieeffizienz.
  \end{solution}
\end{exercise}

\begin{exercise}{Reaktive Systeme}
  Was ist ein reaktives System?

  \begin{solution}
    Ein reaktives System ist ein System, das auf externe Ereignisse reagiert und in Echtzeit agiert.
  \end{solution}
\end{exercise}

\begin{exercise}{Design Flows}
  Wie könnten Design Flows modelliert werden?

  \begin{solution}
    Design Flows können modelliert werden als eine Abfolge von Schritten, die zur Entwicklung eines eingebetteten Systems führen.
  \end{solution}
\end{exercise}

\begin{exercise}{V-Modell}
  Was ist das V-Modell?

  \begin{solution}
    Das V-Modell stellt den einen Entwicklungsprozess und hebt die Verbindung zwischen den Entwicklungsphasen und den Testphasen hervor.
  \end{solution}
\end{exercise}

\begin{exercise}{Synthese}
  Was ist Synthese?

  \begin{solution}
    Synthese ist der Prozess, bei dem ein System aus einer Spezifikation erstellt wird.
  \end{solution}
\end{exercise}

\sheet{Spezifikationen \& Modellierung (Marwedel)}

\begin{exercise}{Modell}
  Was ist ein (Design-)Modell?

  \begin{solution}
    Ein (Design-)Modell ist eine vereinfachte Darstellung eines Systems, die verwendet wird, um das Verhalten und die Struktur des Systems zu beschreiben.
  \end{solution}
\end{exercise}

\begin{exercise}{Anforderungen}
  Nennen Sie 6 Anforderungen an Spezifikations-/Modellierungssprachen für eingebettete Systeme!

  \begin{solution}
    Anforderungen an Spezifikations-/Modellierungssprachen für eingebettete Systeme sind z.B. Echtzeitfähigkeit, Energieeffizienz, Sicherheit, Skalierbarkeit, Wiederverwendbarkeit, und Verständlichkeit.
  \end{solution}
\end{exercise}

\begin{exercise}{Deadlocks}
  Könnte unsere Spezifikation zu Deadlocks führen? Wenn ja, warum?

  \begin{solution}
    Ja, unsere Spezifikation könnte zu Deadlocks führen, wenn z.B. Prozesse auf gegenseitige Exklusivität warten.
  \end{solution}
\end{exercise}

\begin{exercise}{Models of Computation}
  Was ist ein "Model of Computation" (MoC)?

  \begin{solution}
    Ein "Model of Computation" (MoC) ist eine formale Beschreibung der Berechnung, die verwendet wird, um das Verhalten von Systemen zu modellieren.
  \end{solution}
\end{exercise}

\begin{exercise}{Job vs. Task}
  Wie kann man einen Job von einem Task unterscheiden?

  \begin{solution}
    Ein Job ist eine Einheit von Arbeit, die von einem Prozessor ausgeführt wird, während ein Task eine abstrakte Einheit von Arbeit ist, die von einem Prozessor ausgeführt werden kann.
  \end{solution}
\end{exercise}

\begin{exercise}{Kommunikation}
  Besschreiben Sie zwei Schlüsseltechniken für die Kommunikation in Computern!

  \begin{solution}
    Zwei Schlüsseltechniken für die Kommunikation in Computern sind Shared Memory und Message Passing.

    Shared Memory: Shared Memory ist eine Kommunikationstechnik, bei der mehrere Prozesse auf gemeinsame Speicherbereiche zugreifen können.

    Message Passing: Message Passing ist eine Kommunikationstechnik, bei der Prozesse Nachrichten austauschen, um Daten zu übertragen.
  \end{solution}
\end{exercise}

\begin{exercise}{Beschreibung}
  Beschreiben Sie Techniken, die für die Erfassung von Anfangsideen über das zu entwerfende System verwendet werden können!

  \begin{solution}
    Techniken, die für die Erfassung von Anfangsideen über das zu entwerfende System verwendet werden können, sind z.B. Brainstorming, Mind Mapping, und Prototyping.

    Brainstorming: Brainstorming ist eine kreative Technik, bei der Ideen gesammelt und diskutiert werden, um neue Lösungen zu finden.

    Mind Mapping: Mind Mapping ist eine visuelle Technik, bei der Ideen und Konzepte in einem Diagramm dargestellt werden, um Zusammenhänge zu verdeutlichen.

    Prototyping: Prototyping ist eine Technik, bei der ein Modell oder eine Simulation des Systems erstellt wird, um das Design zu überprüfen und zu verbessern.
  \end{solution}
\end{exercise}

\begin{exercise}{Determinanz}
  Sind StateCharts deterministische Modelle, wenn wir der Semantik von StateMate folgen? Und wie schaut es mit SDL aus?

  \begin{solution}
    Ja, StateCharts sind deterministische Modelle, wenn wir der Semantik von StateMate folgen, weil sie eine wohldefinierte Semantik haben, die sicherstellt, dass das Verhalten des Systems eindeutig ist.

    SDL ist ebenfalls ein deterministisches Modell, weil es eine wohldefinierte Semantik hat, die sicherstellt, dass das Verhalten des Systems eindeutig ist.
  \end{solution}
\end{exercise}

\begin{exercise}{Philosophenproblem}
  Wie sieht ein kompaktes Modell des Philosophenproblems aus?

  \begin{solution}
    Ein kompaktes Modell des Philosophenproblems würde die Philosophen als Prozesse und die Gabeln als Ressourcen modellieren, die von den Prozessen gemeinsam genutzt werden.
  \end{solution}
\end{exercise}

\begin{exercise}{Asynchrone Kommunikation}
  Nennen Sie einige Beispiele für asynchrone Kommunikation, sowie für Broadcast Kommunikation!

  \begin{solution}
    Asynchrone Kommunikation: StateCharts, SDL, CSP, MPI

    Broadcast: StateCharts und SDL verwenden einen Broadcast-Mechanismus für die Aktualisierung von Variablen.
  \end{solution}
\end{exercise}

\begin{exercise}{UML Diagramme}
  Welche Diagrammtypen werden von UML unterstützt? Sequence Charts, Record Charts, Y-Charts, Use Cases, Activity Diagrams, Circuit Diagrams?

  \begin{solution}
    UML unterstützt Sequence Charts, Use Cases, und Activity Diagrams, aber keine Record Charts, Y-Charts, oder Circuit Diagrams.
  \end{solution}
\end{exercise}

\sheet{Hardware (Marwedel)}

\begin{exercise}{Signal}
  Was ist ein Signal?

  \begin{solution}
    Ein Signal ist eine physikalische Größe, die Informationen überträgt und verarbeitet.
  \end{solution}
\end{exercise}

\begin{exercise}{Analog vs. Digital}
  Welchen Schaltkreis benötigen wir für den Übergang von kontinuierlicher Zeit zu diskreter Zeit?

  \begin{solution}
    Wir benötigen einen Sample-and-Hold-Schaltkreis für den Übergang von kontinuierlicher Zeit zu diskreter Zeit.
  \end{solution}
\end{exercise}

\begin{exercise}{Abtasttheorem (Sampling Theorem)}
  Was besagt das Abtasttheorem?

  \begin{solution}
    Das Abtasttheorem besagt, dass ein Signal korrekt rekonstruiert werden kann, wenn es mit einer Rate abgetastet wird, die mindestens doppelt so hoch ist wie die höchste Frequenzkomponente des Signals.
  \end{solution}
\end{exercise}

\begin{exercise}{Quantization Noise}
  Was versteht man unter Quantization Noise?

  \begin{solution}
    Quantization Noise ist das Rauschen, das durch die Diskretisierung eines analogen Signals in ein digitales Signal entsteht.
  \end{solution}
\end{exercise}

\begin{exercise}{DSP Processors}
  Was sind die Merkmale von DSP-Prozessoren?

  \begin{solution}
    DSP-Prozessoren sind spezialisierte Prozessoren, die für die Verarbeitung von Signalen optimiert sind und über Funktionen wie Multiplikation, Addition, und Division verfügen.
  \end{solution}
\end{exercise}

\begin{exercise}{FPGAs}
  Aus welchen Komponenten bestehen FPGAs? Welche davon werden verwendet, um Boolesche Funktionen zu implementieren? Wie werden FPGAs konfiguriert? Sind FPGAs energieeffizient? Für welche Anwendungen sind FPGAs gut geeignet?

  \begin{solution}
    FPGAs bestehen aus Logikblöcken, Speicherblöcken und Verbindungen.

    Logikblöcke werden verwendet, um Boolesche Funktionen zu implementieren.

    FPGAs werden mit einem Konfigurations-Bitstrom konfiguriert.

    FPGAs sind energieeffizient und eignen sich gut für Anwendungen, die hohe Leistung und Flexibilität erfordern.
  \end{solution}
\end{exercise}

\begin{exercise}{VLIW Processors}
  Was ist ein VLIW-Prozessor?

  \begin{solution}
    Ein VLIW-Prozessor ist ein Prozessor, der mehrere Instruktionen gleichzeitig ausführt, indem er mehrere Funktionsblöcke parallel betreibt.
  \end{solution}
\end{exercise}

\begin{exercise}{Single-ISA Heterogeneous Multi-Core Architecture}
  Was versteht man unter einer "Single-ISA heterogeneous multi-core architecture"? Welche Vorteile sehen Sie für eine solche Architektur?

  \begin{solution}
    Eine "Single-ISA heterogeneous multi-core architecture" ist eine Architektur, bei der mehrere Prozessoren mit unterschiedlichen Eigenschaften auf einem Chip integriert sind.

    Vorteile einer solchen Architektur sind z.B. die Möglichkeit, verschiedene Anwendungen auf einem Chip auszuführen, die Verbesserung der Leistung und Energieeffizienz, und die Flexibilität bei der Anpassung an unterschiedliche Anforderungen.
  \end{solution}
\end{exercise}

\begin{exercise}{GPU \& MPSoC}
  Was ist der Unterschied zwischen einer GPU und einem MPSoC?

  \begin{solution}
    Eine GPU ist ein Prozessor, der für die Verarbeitung von Grafiken optimiert ist, während ein MPSoC ein Multiprozessorsystem auf einem Chip ist, das für die Verarbeitung von Signalen und Daten optimiert ist.
  \end{solution}
\end{exercise}

\begin{exercise}{Boole'sche Funktionen}
  Einige FPGAs unterstützen die Implementierung aller Booleschen Funktionen von sechs Variablen. Wie viele solcher Funktionen gibt es? Wir ignorieren, dass sich einige Funktionen nur durch eine Umbenennung von Variablen unterscheiden.

  \begin{solution}
    Es gibt $2^{2^6} = 2^{64}$ Boolesche Funktionen von sechs Variablen.
  \end{solution}
\end{exercise}

\begin{exercise}{Memory Hierarchy}
  Im Kontext von Speichern sagen wir manchmal "klein ist schön". Was könnte der Grund dafür sein?

  \begin{solution}
    Der Grund dafür ist, dass kleinere Speicher schneller und energieeffizienter sind als größere Speicher.
  \end{solution}
\end{exercise}

\begin{exercise}{Memory Hierarchy}
  Einige Ebenen der Speicherhierarchie können dem Anwendungsprogrammierer verborgen sein. Warum sollte sich ein solcher Programmierer dennoch für die Architektur solcher Ebenen interessieren?

  \begin{solution}
    Ein Programmierer sollte sich für die Architektur solcher Ebenen interessieren, um die Leistung und Energieeffizienz seines Programms zu optimieren.
  \end{solution}
\end{exercise}

\begin{exercise}{Scratchpad Memory}
  Was ist Scratchpad Memory? Wie können wir sicherstellen, dass ein bestimmtes Speicherobjekt im Scratchpad Memory gespeichert wird?

  \begin{solution}
    Scratchpad Memory ist ein kleiner, schneller und energieeffizienter Speicher, der zur Speicherung häufig verwendeter Daten verwendet wird.

    Um sicherzustellen, dass ein bestimmtes Speicherobjekt im Scratchpad Memory gespeichert wird, können wir den
  \end{solution}
\end{exercise}

\begin{exercise}{Techniken zur Nutzung von Scratchpad Memory}
  Geben Sie einen Überblick über Techniken, um Scratch Pad Memory zu nutzen!

  \begin{solution}
    \begin{itemize}
      \item Statische Analyse: Statische Analyse wird verwendet, um die Größe und den Zugriff auf den Scratchpad-Speicher zu bestimmen.
      \item Compiler-Optimierungen: Compiler-Optimierungen werden verwendet, um den Zugriff auf den Scratchpad-Speicher zu optimieren und die Speichernutzung zu maximieren.
      \item Profiling: Profiling wird verwendet, um das Verhalten des Systems zu analysieren und den Scratchpad-Speicher effizient zu nutzen.
    \end{itemize}
  \end{solution}
\end{exercise}

\begin{exercise}{Energieffizienz}
  Wie unterscheiden sich verschiedene Hardware-Technologien hinsichtlich ihrer Energieeffizienz?

  \begin{solution}
    Eingebette Systeme sind zumeist energieeffizienter als herkömmliche Computer, da sie speziell für die Anforderungen von eingebetteten Systemen optimiert sind.
  \end{solution}
\end{exercise}

\begin{exercise}{Optimierunng}
  Vergleichen Sie die Energieeffizienz von verschiedenen Technologien zur Verarbeitung von Informationen in eingebetteten Systemen!

  \begin{solution}
    FPGAs sind energieeffizienter als GPUs und MPSoCs, da sie speziell für die Verarbeitung von Signalen und Daten optimiert sind.
  \end{solution}
\end{exercise}

\begin{exercise}{Battery Life}
  Angenommen, Ihr Mobiltelefon nutzt eine Lithiumbatterie mit einer Kapazität von 720mAh. Die Nennspannung der Batterie beträgt 3,7V. Unter der Annahme eines konstanten Stromverbrauchs von 1W, wie lange würde es dauern, die Batterie zu entleeren? Alle sekundären Effekte wie abnehmende Spannungen sollten bei dieser Berechnung ignoriert werden.

  \begin{solution}
    Die Leistungsaufnahme von 1W entspricht einem Strom von $1W / 3.7V = 0.27A$. Die Batteriekapazität von 720mAh entspricht einer Ladung von $720mAh = 0.72Ah$. Die Zeit, die es dauern würde, die Batterie zu entleeren, beträgt $0.72Ah / 0.27A = 2.67h$.
  \end{solution}
\end{exercise}

\begin{exercise}{Sicherheit}
  Welche Herausforderungen bestehen für die Sicherheit von eingebetteten Systemen?

  \begin{solution}
    Herausforderungen für die Sicherheit von eingebetteten Systemen sind z.B. die Anfälligkeit für Angriffe, die Komplexität der Systeme, und die Notwendigkeit, die Integrität und Vertraulichkeit der Daten zu schützen.
  \end{solution}
\end{exercise}

\begin{exercise}{Side-Channel Attacks}
  Was ist ein "Side-Channel-Angriff"? Bitte geben Sie Beispiele für Side-Channel-Angriffe!

  \begin{solution}
    Ein "Side-Channel-Angriff" ist ein Angriff, bei dem ein Angreifer Informationen über ein System aus den physikalischen Eigenschaften des Systems gewinnt.

    Beispiele für Side-Channel-Angriffe sind z.B. Timing-Angriffe, Stromverbrauchsanalysen, und elektromagnetische Strahlung.
  \end{solution}
\end{exercise}

\sheet{Software (Marwedel)}

\begin{exercise}{Embedded OS}
  Welche Anforderungen müssen für ein eingebettetes Betriebssystem erfüllt sein?

  \begin{solution}
    Anforderungen für ein eingebettetes Betriebssystem sind z.B. Echtzeitfähigkeit, geringer Speicherbedarf, und Energieeffizienz.
  \end{solution}
\end{exercise}

\begin{exercise}{Anpassungsfähigkeit}
  Welche Techniken können verwendet werden, um ein eingebettetes Betriebssystem auf die erforderliche Weise anzupassen?

  \begin{solution}
    Häufig verwendet werden Techniken wie Konfigurationsoptionen, Modulare Architektur, und Anpassung der Betriebssystemkomponenten.
  \end{solution}
\end{exercise}

\begin{exercise}{Real-Time Operating System}
  Welche Anforderungen müssen für ein Echtzeit-Betriebssystem erfüllt sein? Wie unterscheiden sie sich von den Anforderungen eines Standard-Betriebssystems? Welche Funktionen eines Standard-Betriebssystems wie Windows oder Linux könnten in einem RTOS fehlen?

  \begin{solution}
    Anforderungen für ein Echtzeit-Betriebssystem sind z.B. deterministische Ausführungszeiten, Echtzeitfähigkeit, und geringer Speicherbedarf.

    Echtzeit-Betriebssysteme unterscheiden sich von Standard-Betriebssystemen durch ihre Echtzeitfähigkeit und ihre geringen Speicheranforderungen. Funktionen eines Standard-Betriebssystems wie Windows oder Linux, die in einem RTOS fehlen könnten, sind z.B. Multitasking, Dateisysteme, und Netzwerkunterstützung.
  \end{solution}
\end{exercise}

\begin{exercise}{Priority Inversion}
  Geben Sie ein Beispiel, das die Prioritätsinversion für ein System mit drei Jobs zeigt! Wie kann sich die Prioritätsinversion auf das System auswirken?

  \begin{solution}
    Ein Beispiel für die Prioritätsinversion in einem System mit drei Jobs wäre, wenn ein niedrig priorisierter Job eine Ressource blockiert, die von einem höher priorisierten Job benötigt wird.

    Die Prioritätsinversion kann dazu führen, dass der höher priorisierte Job blockiert wird und nicht rechtzeitig ausgeführt werden kann, was zu einer Verletzung der Echtzeitanforderungen führen kann.
  \end{solution}
\end{exercise}

\begin{exercise}{Deadlock Prevention}
  Welche Ressourcenzugriffsprotokolle verhindern Deadlocks, die durch exklusiven Zugriff auf Ressourcen verursacht werden?

  \begin{solution}
    PCP: Das Protokoll zur Prävention von zyklischen Wartebedingungen (PCP) verhindert Deadlocks, die durch exklusiven Zugriff auf Ressourcen verursacht werden, indem es sicherstellt, dass Prozesse nur auf Ressourcen zugreifen, die sie benötigen.

    Banker's Algorithm: Der Banker's Algorithmus verhindert Deadlocks, die durch exklusiven Zugriff auf Ressourcen verursacht werden, indem er sicherstellt, dass Prozesse nur auf Ressourcen zugreifen, die verfügbar sind.

    Wait-Die: Das Wait-Die-Protokoll verhindert Deadlocks, die durch exklusiven Zugriff auf Ressourcen verursacht werden, indem es Prozesse in Wartezustände versetzt, wenn sie auf Ressourcen warten, die von anderen Prozessen verwendet werden.
  \end{solution}
\end{exercise}

\begin{exercise}{Linux als OS}
  Welche Probleme müssen gelöst werden, wenn Linux als Betriebssystem für ein eingebettetes System verwendet wird?

  \begin{solution}
    Probleme, die gelöst werden müssen, wenn Linux als Betriebssystem für ein eingebettetes System verwendet wird, sind z.B. die Anpassung an die spezifischen Anforderungen des Systems, die Reduzierung des Speicherbedarfs, und die Optimierung der Energieeffizienz.
  \end{solution}
\end{exercise}

\sheet{Weitere Kapitel (Marwedel)}

\begin{exercise}{Evaluation \& Validation}
  Welche Bedingungen müssen bei der Berechnung der WCET erfüllt sein?

  \begin{solution}
    Bedingungen, die bei der Berechnung der WCET erfüllt sein müssen, sind z.B. deterministische Ausführungszeiten, keine Interrupts, und keine Caches.
  \end{solution}
\end{exercise}

\begin{exercise}{Thermal Power}
  Consider a copper plate of area $A=10cm^2$ and length $5mm$. How much thermal power is transferred if the difference between the temperatures at the two ends of the plate is $10^\circ C$?

  \begin{solution}
    The thermal power transferred through a copper plate can be calculated using the formula $P = k \cdot A \cdot \Delta T / L$, where $k$ is the thermal conductivity of copper, $A$ is the area of the plate, $\Delta T$ is the temperature difference, and $L$ is the length of the plate.
  \end{solution}
\end{exercise}

\begin{exercise}{Exponential Distribution}
  Consider a hard disk drive for which we assume that half of the drives have failed after 5000 hours of operation. Let us assume that failures follow an exponential distribution. Compute the corresponding value of $\lambda$!

  \begin{solution}
    The value of $\lambda$ for a hard disk drive that has a failure rate of 50\% after 5000 hours of operation can be computed using the formula $\lambda = -\ln(0.5) / 5000$.
  \end{solution}
\end{exercise}

\begin{exercise}{Job Scheduling}
  Suppose that we have a set of four jobs. Release times $r_i$, deadlines $D_i$, and execution times $C_i$ are as follows:
  \begin{itemize}
    \item $J1$: $r_1=10$, $D_1=18$, $C_1=4$
    \item $J2$: $r_2=0$, $D_2=28$, $C_2=12$
    \item $J3$: $r_3=6$, $D_3=17$, $C_3=3$
    \item $J4$: $r_4=3$, $D_4=13$, $C_4=6$
  \end{itemize}
  Generate a graphical representation of schedules for this job set, using earliest deadline first (EDF) and least laxity (LL) scheduling algorithms! For LL scheduling, indicate laxities for all jobs at all context switch times. Will any job miss its deadline?

  \begin{solution}
    The graphical representation of schedules for the job set using EDF and LL scheduling algorithms is as follows:
    \begin{itemize}
      \item EDF: $J2 \rightarrow J4 \rightarrow J3 \rightarrow J1$
      \item LL: $J2 \rightarrow J4 \rightarrow J3 \rightarrow J1$
    \end{itemize}
    No job will miss its deadline.
  \end{solution}
\end{exercise}

\begin{exercise}{Rate Monotonic Scheduling}
  Suppose that we have a system comprising two tasks. Task 1 has a period of 5 and an execution time of 2. The second task has a period of 7 and an execution time of 4. Let the deadlines be equal to the periods. Assume that we are using rate monotonic scheduling (RMS). Could any of the two tasks miss its deadline, due to a too high processor utilization? Compute this utilization, and compare it to a bound which would guarantee schedulability! Generate a graphical representation of the resulting schedule! Suppose that tasks will always run to their completion, even if they missed their deadline.

  \begin{solution}
    The utilization bound for rate monotonic scheduling is $U = n(2^{1/n} - 1)$, where $n$ is the number of tasks. For two tasks, the utilization bound is $U = 0.8284$. The utilization of the task set is $U = 0.6$, which is less than the utilization bound, so no task will miss its deadline.
  \end{solution}
\end{exercise}

\begin{exercise}{Earliest Deadline First Scheduling}
  Consider the same task set as in the previous assignment. Use earliest deadline first (EDF) for scheduling. Can any of the tasks miss its deadline? If not, why not? Generate a graphical representation of the resulting schedule!

  \begin{solution}
    No task will miss its deadline with earliest deadline first (EDF) scheduling because EDF is optimal for scheduling real-time tasks.
  \end{solution}
\end{exercise}

\begin{exercise}{Loop Unrolling}
  Loop Unrolling ist eine der potenziell nützlichen Optimierungen. Nennen Sie bitte zwei potenzielle Vorteile und zwei potenzielle Probleme!

  \begin{solution}
    Potenzielle Vorteile von Loop Unrolling sind die Reduzierung des Overheads von Schleifen und die Erhöhung der Parallelität von Instruktionen.

    Potenzielle Probleme von Loop Unrolling sind die Erhöhung des Speicherbedarfs und die Erhöhung der Codegröße.
  \end{solution}
\end{exercise}

\begin{exercise}{Loop Tiling}
  Angenommen, Sie möchten Loop Tiling verwenden. Wie können Sie das Tiling an die vorliegende Speicherarchitektur anpassen?

  \begin{solution}
    Loop Tiling kann an die vorliegende Speicherarchitektur angepasst werden, indem die Größe der Tiles so gewählt wird, dass sie in den Cache passen und die Zugriffe auf den Speicher optimiert werden.
  \end{solution}
\end{exercise}

\begin{exercise}{Festkomma Arithmetik}
  Für welche Architekturen würden Sie die größten Vorteile von einem Austausch von Gleitkomma-Arithmetik durch Festkomma-Arithmetik erwarten?

  \begin{solution}
    Die größten Vorteile von einem Austausch von Gleitkomma-Arithmetik durch Festkomma-Arithmetik würden Sie für Architekturen erwarten, die auf eingebetteten Systemen und Signalverarbeitungsaufgaben basieren, aufgrund der geringeren Komplexität und des geringeren Speicherbedarfs von Festkomma-Arithmetik.
  \end{solution}
\end{exercise}

\end{document}