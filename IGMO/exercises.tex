\documentclass{article}

\usepackage[solutions]{xrcise}

\subject{Informationsgestützte Modellierung von Organisationen}
\semester{Summer 2024}
\author{Leopold Lemmermann}

\begin{document}\createtitle

\sheet{Vorlesungsfragen}
\begin{exercise}{Einführung}
  \begin{enumerate}
    \item Was ist ein Modell?
    \item Welche typischen Eigenschaften weisen Modelle auf?
    \item Zu welchen Zwecken werden Modelle gebildet?
    \item Nennen Sie jeweils ein Beispiel für Komplexitätsreduzierung durch Idealisierung und durch Abstraktion.
    \item Worin unterscheiden sich Simulationsmodelle und analytische Modelle? Geben Sie zwei Merkmale an.
    \item Worin unterscheiden sich Simulationen mit wissenschaftlichem Anspruch und Computerspiele des Genres "Simulation"? Nennen Sie zwei Merkmale.
  \end{enumerate}

  \begin{solution}
    \begin{enumerate}
      \item Ein Modell ist eine vereinfachte Darstellung eines Systems.
      \item Modelle sind abstrakt, vereinfacht, formalisiert und haben eine bestimmte Perspektive.
      \item Modelle werden gebildet, um komplexe Sachverhalte zu verstehen, zu analysieren, zu beschreiben, zu prognostizieren oder zu simulieren.
      \item Komplexitätsreduzierung durch Idealisierung: Ein Modell eines Autos kann als Punktmasse betrachtet werden.
      \item Komplexitätsreduzierung durch Abstraktion: Ein Modell eines Autos kann als Kasten betrachtet werden.
      \item Simulationen mit wissenschaftlichem Anspruch sind wissenschaftlich validiert und reproduzierbar. Computerspiele des Genres "Simulation" sind unterhaltend und nicht wissenschaftlich validiert.
    \end{enumerate}
  \end{solution}
\end{exercise}

\begin{exercise}{Diskrete Modellierung}
  \begin{enumerate}
    \item Beschreiben Sie das grundsätzliche Ablaufschema der ereignisdiskreten Simulation!
    \item Wann werden Simulationsläufe beendet, d.h. welche typischen Abbruchbedingungen gibt es in der ereignisdiskreten Simulation?
    \item Beschreiben Sie das grundsätzliche Vorgehen zum Entwurf eines prozessorientierten Simulationsmodells!
    \item Vergleichen Sie ereignis- und prozessorientierte Modellierung! Stellen Sie insbesondere wesentliche Unterschiede dar!
    \item Begründen Sie, warum für Lagerhaltungsmodelle typischerweise der ereignisorientierte Modellierungsstil besser geeignet ist als der prozessorientierte Modellierungsstil.
    \item Wie modelliert man wiederholte Ankünfte…
          \begin{enumerate}
            \item …im ereignisorientierten Ansatz?
            \item …im prozessorientierten Ansatz?
          \end{enumerate}
    \item Skizzieren Sie das BPMN-Prozessdiagramm eines typischen Kunden-Prozesses z.B. Job in einem Produktionssystem
    \item Nennen Sie drei typischerweise in der Simulation verwandte Typen von BPMN-Zwischenereignissen und beschreiben Sie ihre Semantik.
    \item Warum sind für die BPMN-Prozesssimulation implizit erzeugte Warteschlangen notwendig?
    \item Wie kann man in BPMN-Prozessdiagrammen die begrenzte Wartebereitschaft eines Kunden abbilden?
  \end{enumerate}

  \begin{solution}
    \begin{enumerate}
      \item Ereignisdiskrete Simulation: Initialisierung, Ereignisverarbeitung, Aktualisierung der Systemzustände, Terminierung.
      \item Abbruchbedingungen: Erreichen einer maximalen Simulationszeit, Erreichen einer maximalen Anzahl an Ereignissen, Erreichen einer maximalen Anzahl an Kunden.
      \item Prozessorientierte Modellierung: Identifikation der Prozesse, Identifikation der Ressourcen, Identifikation der Ereignisse, Identifikation der Warteschlangen.
      \item Unterschiede: Ereignisorientierte Modellierung: Fokus auf Ereignisse, diskrete Zeit, keine explizite Repräsentation von Ressourcen. Prozessorientierte Modellierung: Fokus auf Prozesse, kontinuierliche Zeit, explizite Repräsentation von Ressourcen.
      \item Lagerhaltungsmodelle: Ereignisorientierte Modellierung ist besser geeignet, da die Ereignisse (z.B. Bestellungen) diskret sind und die Zeit zwischen den Ereignissen nicht von der Anzahl der Ressourcen abhängt.
      \item Wiederholte Ankünfte: Ereignisorientierter Ansatz: Ankunft eines Kunden, Ankunft eines weiteren Kunden, … Prozessorientierter Ansatz: Ankunft eines Kunden, Verarbeitung des Kunden, Ankunft eines weiteren Kunden, Verarbeitung des weiteren Kunden, …
      \item BPMN-Prozessdiagramm: Startereignis, Aktivität, Entscheidung, Verzweigung, Zusammenführung, Endereignis.
      \item Typen von BPMN-Zwischenereignissen: Zeitsteuerung (z.B. Timer-Ereignis), Nachrichtenereignis, Signalereignis.
      \item Implizit erzeugte Warteschlangen: Notwendig, um die Reihenfolge der Aktivitäten zu steuern und Engpässe zu erkennen.
      \item Begrenzte Wartebereitschaft: Durch Timer-Ereignisse oder Signalereignisse.
    \end{enumerate}
  \end{solution}
\end{exercise}

\begin{exercise}{Simulationssoftware}
  \begin{enumerate}
    \item Welche Arten von Simulationssoftware kann man unterscheiden?
    \item Nennen Sie jeweils zwei Vor- und Nachteile von Simulationssoftware auf Werkzeugebene.
    \item Welche simulationsspezifischen Auswahlkriterien für Simulationssoftware kennen Sie?
    \item Beschreiben Sie Potential und Grenzen von Animation in der Simulation.
    \item Beschreiben die den Zusammenhang zwischen konzeptuellem Modell \& Computermodell einer Simulationsstudie mit IYOPRO.
    \item Inwiefern unterstützt IYOPRO die Modellierung und Simulation von Ressourcen?
    \item Gehen Sie insbesondere auf die möglichen Zustände einer Ressource ein.
    \item Worin unterscheiden sich Blackbox- und Whitebox-Komponenten in Software-Frameworks? Nennen sie jeweils ein Beispiel in DESMO-J!
  \end{enumerate}

  \begin{solution}
    \begin{enumerate}
      \item Arten von Simulationssoftware: Allgemeine Programmiersprachen, Simulationssoftware auf Werkzeugebene, Simulationssoftware auf Anwendungsebene.
      \item Vor- und Nachteile von Simulationssoftware auf Werkzeugebene: Vorteile: Einfache Bedienung, Schnelligkeit. Nachteile: Eingeschränkte Flexibilität, geringe Leistungsfähigkeit.
      \item Auswahlkriterien für Simulationssoftware: Kosten, Funktionsumfang, Benutzerfreundlichkeit, Skalierbarkeit, Dokumentation.
      \item Potential und Grenzen von Animation: Potential: Veranschaulichung, Verständnis. Grenzen: Komplexität, Ablenkung.
      \item Konzeptuelles Modell \& Computermodell: Konzeptuelles Modell: Abstrakte Beschreibung des Systems. Computermodell: Implementierung des konzeptuellen Modells in einer Simulationssoftware.
      \item IYOPRO: Modellierung und Simulation von Ressourcen: Zustände: Frei, Belegt, Wartend, Defekt.
      \item Blackbox- und Whitebox-Komponenten: Blackbox: Kein Zugriff auf interne Zustände, z.B. Warteschlange. Whitebox: Zugriff auf interne Zustände, z.B. Ressource.
    \end{enumerate}
  \end{solution}
\end{exercise}

\begin{exercise}{Simulationsstatistik \& Optimierung}
  \begin{enumerate}
    \item Warum sollte man Simulationsläufe mehrfach wiederholen?
    \item Wie kann man Pseudozufallszahlen erzeugen, die im Intervall [0,1) näherungsweise stetig gleichverteilt sind?
    \item Was versteht man unter einem Konfidenzintervall?
    \item Welche Bestandteile gehören zu einem Optimierungsproblem?
    \item Erläutern Sie die grundsätzliche Vorgehensweise eines genetischen Algorithmus!
  \end{enumerate}

  \begin{solution}
    \begin{enumerate}
      \item Wiederholung von Simulationsläufen: Reduktion von Zufallseinflüssen, Verbesserung der Aussagekraft.
      \item Pseudozufallszahlen: Linearer Kongruenzgenerator, Mersenne-Twister.
      \item Konfidenzintervall: Intervall, in dem ein Schätzwert mit einer bestimmten Wahrscheinlichkeit liegt.
      \item Bestandteile eines Optimierungsproblems: Zielfunktion, Entscheidungsvariablen, Nebenbedingungen.
      \item Genetischer Algorithmus: Initialisierung, Selektion, Rekombination, Mutation, Evaluation, Ersetzung.
    \end{enumerate}
  \end{solution}
\end{exercise}

\begin{exercise}{Simulationspraxis}
  \begin{enumerate}
    \item "Wir müssen unsere Logistik verbessern!" - Welche Ziele versteht man hierunter üblicherweise in Unternehmen?
    \item Was spricht für die Beauftragung externer Simulationsdienstleister, um eine betriebliche Simulationsstudie durchzuführen? Was spricht gegen Sie? Nennen Sie jeweils zwei Faktoren!
    \item Welche Rollen kann man bei Durchführung einer Simulationsstudie unterscheiden?
    \item Nennen Sie drei typische Fehlerquellen in einer Simulationsstudie!
  \end{enumerate}

  \begin{solution}
    \begin{enumerate}
      \item Ziele der Logistikverbesserung: Kostensenkung, Effizienzsteigerung, Lieferzeitverkürzung.
      \item Externe Simulationsdienstleister: Pro: Expertise, Unabhängigkeit. Contra: Kosten, fehlendes Detailwissen.
      \item Rollen bei einer Simulationsstudie: Auftraggeber, Projektleiter, Simulationsdienstleister, Anwender.
      \item Fehlerquellen in einer Simulationsstudie: Falsche Annahmen, Modellfehler, Implementierungsfehler.
    \end{enumerate}
  \end{solution}
\end{exercise}

\begin{exercise}{(IT-)Organisation \& Prozesse}
  \begin{enumerate}
    \item Setzen Sie Porters Wertschöpfungskette in Beziehung zu den beschriebenen Unternehmensaktivitäten!
    \item Erklären Sie Sinn und Zweck von Modellierung und leiten Sie daraus Vor- und Nachteile des modellbasierten Problemlösens ab!
    \item Modellieren Sie einen gegebenen Geschäftsprozess und schätzen Sie die Performanz eines Prozesses anhand qualitativer sowie quantitativer Analyseverfahren ab!
    \item Beschreiben Sie Methoden und Strategien des Komplexitätsmanagements!
    \item Erklären Sie, weshalb es so vielfältige Aufbauten von Organisationen gibt und verstehen Sie, welche Faktoren einen Einfluss auf das Design von Organisationen haben!
    \item Erklären Sie den Sinn und Zweck der Referenzmodellierung und leiten Sie daraus Vor- und Nachteile ab!
    \item Welches sind verschiedene Koordinationsmechanismen in Organisationen und wie können diese in unterschiedlichen Aufbauten von Organisationen realisiert werden?
    \item Was ist organisatorische Ambidextrie und welche Designs sind förderlich für ein ambidextrielles Konstrukt?
    \item Was ist der Unterschied zwischen "agil sein" und "agil anwenden" ("being agile" vs. "doing agile") und welche verschiedenen Arten der Agilität gibt es?
    \item Beschreiben Sie den grundsätzlichen Aufbau klassischer IT-Funktionen und wie diese in Organisationen positioniert sind!
    \item Welches sind die verschiedenen Rollen und Aufgabenbereiche innerhalb einer IT-Funktion?
    \item Warum ist die veränderte Rolle der IT-Funktion im Rahmen der digitalen Transformation wichtig und wieso ist das Business-IT Alignment von Bedeutung?
  \end{enumerate}

  \begin{solution}
    \begin{enumerate}
      \item Porters Wertschöpfungskette: Primäre Aktivitäten (Eingangslogistik, Produktion, Ausgangslogistik, Marketing, Vertrieb, Service) und Sekundäre Aktivitäten (Unternehmensinfrastruktur, Personalwirtschaft, Technologieentwicklung, Beschaffung).
      \item Sinn und Zweck von Modellierung: Verständnis, Analyse, Prognose, Simulation. Vor- und Nachteile: Vorteile: Strukturierung, Visualisierung. Nachteile: Vereinfachung, Abstraktion.
      \item Geschäftsprozessmodellierung: Modellierung eines Geschäftsprozesses, Analyse der Performanz. Qualitative Analyse: Schwachstellenanalyse, Verbesserungsvorschläge. Quantitative Analyse: Simulation, Optimierung.
      \item Komplexitätsmanagement: Methoden: Modularisierung, Standardisierung, Prozessoptimierung. Strategien: Reduktion, Beherrschung, Nutzung.
      \item Aufbau von Organisationen: Faktoren: Größe, Branche, Umfeld, Strategie. Vielfältige Aufbauten: Funktionale Organisation, Divisionale Organisation, Matrixorganisation.
      \item Referenzmodellierung: Sinn und Zweck: Standardisierung, Wiederverwendung. Vor- und Nachteile: Vorteile: Effizienz, Qualität. Nachteile: Anpassungsaufwand, Abhängigkeit.
      \item Koordinationsmechanismen: Markt, Hierarchie, Clan, Adhocratie. Aufbauten von Organisationen: Funktionale Organisation, Divisionale Organisation, Matrixorganisation.
      \item Organisatorische Ambidextrie: Sinn und Zweck: Exploration, Exploitation. Förderliche Designs: Divisionale Organisation, Matrixorganisation.
      \item Agilität: Agil sein: Kultur, Werte. Agil anwenden: Methoden, Prozesse. Arten der Agilität: Strukturelle Agilität, Prozessuale Agilität, Kulturelle Agilität.
      \item IT-Funktionen: Aufbau: IT-Strategie, IT-Entwicklung, IT-Betrieb. Positionierung: Zentral, Dezentral, Gemischt.
      \item Rollen und Aufgabenbereiche: IT-Strategie: CIO, IT-Entwicklung: IT-Architekt, IT-Betrieb: IT-Administrator.
      \item Digitale Transformation: Wandel der IT-Funktion, Business-IT Alignment: Abstimmung von IT und Fachbereich.
    \end{enumerate}
  \end{solution}
\end{exercise}

\begin{exercise}{Komplexe Informationssysteme}
  \begin{enumerate}
    \item Was sind die drei Prinzipien für die Entwicklung soziotechnischer Systeme und warum sind sie wichtig in der IT-gestützten Modellierung und Konzeption von (IT-)Organisationen?
    \item Was sind komplexe Informationssysteme? Benennen Sie Ressourcen, Funktionen und Varianten der Nutzung!
  \end{enumerate}

  \begin{solution}
    \begin{enumerate}
      \item Prinzipien für soziotechnische Systeme: Mensch-Maschine-Interaktion, Mensch-Mensch-Interaktion, Mensch-Organisation-Interaktion. Wichtig: Berücksichtigung der sozialen und organisatorischen Aspekte.
      \item Komplexe Informationssysteme: Ressourcen: Hardware, Software, Daten, Personal. Funktionen: Verarbeitung, Speicherung, Kommunikation. Varianten der Nutzung: Einzelplatzsystem, Mehrplatzsystem, Verteiltes System.
    \end{enumerate}
  \end{solution}
\end{exercise}

\begin{exercise}{Digital Leadership \& Digital Business Strategy}
  \begin{enumerate}
    \item
  \end{enumerate}

  \begin{solution}
    \begin{enumerate}
      \item
    \end{enumerate}
  \end{solution}
\end{exercise}


\sheet[2024]{Probeklausur}
\begin{exercise}{Ereignismodellierung}
  Gegeben sei folgende Modellbeschreibung:
  \par Bei Verwendung der digitalen Währung "ByteCoin" kommunizieren alle teilnehmenden Nutzer und Server miteinander über ein bestimmtes Protokoll:
  \par Bei jedem Nutzer entsteht im Abstand von durchschnittlich drei Tagen (exponentialverteilt) der Wunsch, eine Transaktion, also z.B. eine Überweisung, durchzuführen: Er sendet diese Transaktion über das Netzwerk an alle Server ("Broadcast", das Absenden ist nahezu zeitverzugslos). Die Zeit, bis diese Transaktion über das Netzwerk einen bestimmten Server erreicht, dauert im Mittel 8 Sekunden (normalverteilt, Standardabweichung 2 Sekunden, für jeden Server eine individuelle Zeit). Ein Server, der eine solche Transaktion erhält, speichert diese Transaktion als "wartende Transaktion" ab.
  \par Alle Server betreiben permanent einen als "Mining" bezeichneten Rechenvorgang, das heißt, sie versuchen, die Ihnen vor liegenden wartenden Transaktionen in einen neuen "Transaktionsblock" (im Folgenden nur kurz "Block") zusammenzufassen. Hierfür muss ein rechenintensives mathematisches Problem gelöst werden, dessen hohe Schwierigkeit dafür sorgt, dass es im Mittel nur alle 10 Minuten (exponentialverteilt) einem der Server gelingt, einen Block zu erzeugen. Ein solcher Block ist eine Datenstruktur, die neben allen dem Server bekannten "wartenden Transaktionen" auch einen Index i enthält, um deutlich zu machen, dass dieser Block der Nachfolger eines Vorgängerblocks mit Index i-1 ist. Ein Verweis auf den Vorgängerblock in Form eines Hash-Wertes ist ebenfalls im neuen Block enthalten.
  \par Wenn ein Server einen Block erzeugt hat, darf er sich als "Belohnung" 12,5 ByteCoins gutschreiben. Er löscht die im Block enthaltenen Transaktionen aus seinen "wartenden Transaktionen" und sendet den Block an alle anderen Server des Netzwerks ("Broadcast", nahezu zeitverzugslos). Die Zeit, bis ein anderer Server den Block über das Netzwerk erhalten hat, dauert im Mittel 18 Sekunden (normalverteilt, Standardabweichung 3 Sekunden, für jeden Server eine individuelle Zeit).
  \par Ein Server, der einen Nachfolger zum Block i-1 sucht und dann einen Block mit Index i von einem anderen Server erhält, wird den Block speichern, die dort enthaltenen Transaktionen aus seinen "wartenden Transaktionen" löschen und ab sofort versuchen, einen Nachfolger für diesen Block zu finden.

  \textbf{Splits}

  Als ein Sonderfall besteht im Rahmen der oben beschriebenen Logik die Möglichkeit, dass zwei Server nahezu zeitgleich jeweils einen Block mit Index i erzeugen und versenden, ohne dass sie wegen der Zeitdauer der Block-Übertragung durch das Netzwerk bisher von der Block-Erzeugung des jeweils anderen Servers Kenntnis haben.
  \par Beide Blöcke mit Index i stellen dann jeweils gültige Nachfolger des Vorgängerblocks mit Index i-1 dar und werden an alle Server versandt. So entsteht ein "Split" des Netzwerks: Jeder Server verwendet denjenigen Block mit Index i für das "Mining", den er zuerst erhalten hat und ignoriert mögliche spätere Ankünfte anderer Blöcke mit gleichem Index i. Erst der (spätere) Erhalt von Blöcken, die einen höheren Index aufweisen als alle bisher selbst erzeugten oder von anderen Server erhaltenen Blöcke, würde dazu führen, den neu erhaltenen Block für das künftige Mining zu verwenden (und zwar auch dann, wenn dieser Block kein Nachfolger des im Moment für das Mining verwendeten Blocks ist).

  Bearbeiten Sie hierzu folgende Teilaufgaben.
  \begin{enumerate}
    \item\label{itm:prim} Nennen Sie drei Beispiele für primär interessierende Leistungsgrößen, die eine Simulation des Modells untersuchen könnte.
    \item\label{itm:empf} Würden Sie für die Durchführung einer solchen Simulationsstudie eine ereignis- oder prozessorientierte Modellierung empfehlen? Geben Sie zwei Gründe für Ihre Empfehlung an.
    \item\label{itm:ent} Unabhängig von Ihrer Empfehlung in \ref{itm:empf} soll im Rahmen der Simulationsstudie eine ereignisorientierte Modellierung erstellt werden. Benennen Sie zunächst die zu modellierenden Entitäten und Ereignistypen sowie eventuell für Synchronisationszwecke nützliche Warteschlangen.
    \item\label{itm:mod} Geben Sie eine semi-formale Modellierung der in \ref{itm:ent} identifizierten Ereignistypen an. Verwenden Sie hierfür Flussdiagramme analog zur Vorlesung und Übung.
  \end{enumerate}
  \hint{
    \begin{itemize}
      \item Vergessen sie nicht, anzugeben, auf welchen Entitätstyp bzw. auf welche Entitätstypen sich das Ereignis bezieht, sofern zutreffend.
      \item Sie könnten optional zur Bezeichnung des Ereignisses auch ein Kürzel definieren, so dass bei Verweis auf dieses Ereignis an anderer Stelle Schreibarbeit gespart wird. Damit könnte die Definition eines Ereignisses etwa so beginnen: Ankunft eines LKW an der Kiesgrube (AK) Bezug: LKW
      \item Verweisen Sie explizit auf eventuell benötigte Warteschlangen, etwa Entferne den ersten LKW aus Lade-WS, wobei Sie den Kurz-Bezeichner der Warteschlange (hier: "Lade-WS") unverändert lassen, wenn Sie an anderer Stelle ebenfalls auf diese Warteschlange zugreifen.
      \item Berücksichtigen Sie Versendung neuer Transaktionen sowie das Erzeugen weiterer Blocks im Zeitverlauf.
      \item Falls Sie globale Variablen benötigen, geben Sie zusätzlich zu den Flussdiagrammen eine kurze Definition und den Startwert an, zum Beispiel so: b = 0 / Anzahl der bisher gefundenen Blocks
      \item Wo die Modellbeschreibung nicht ausreichend präzise oder unvollständig ist, dürfen sinnvolle Annahmen getroffen werden. Nicht explizit in der Modellbeschreibung mit einer bestimmten Dauer erwähnte Aktivitäten dürfen als zeitverzugslos angenommen werden.
      \item Vergleichen Sie nach Abschluss der Aufgabe Ihre Antworten zu \ref{itm:ent} und \ref{itm:mod}. Insbesondere: Falls Sie während der Bearbeitung von \ref{itm:mod} Korrekturbedarf bezüglich in der \ref{itm:ent} benannten Entitäten, Ereignistypen oder Warteschlangen festgestellt haben, dann ändern Sie auch Ihre Antwort von \ref{itm:ent} entsprechend!
    \end{itemize}
  }

  \begin{solution}
    \begin{enumerate}
      \item Primär interessierende Leistungsgrößen: Durchschnittliche Zeit bis zur Erzeugung eines Blocks, Anzahl der Splits, Anzahl der erzeugten Blocks.
      \item Empfehlung: Ereignisorientierte Modellierung. Gründe: Diskrete Ereignisse, keine explizite Repräsentation von Ressourcen.
      \item Entitäten und Ereignistypen: Entitäten: Nutzer, Server, Transaktion, Block. Ereignistypen: Ankunft einer Transaktion, Erzeugung eines Blocks, Erhalt eines Blocks.
      \item Modellierung: Flussdiagramme für Ereignistypen
    \end{enumerate}
  \end{solution}
\end{exercise}

\begin{exercise}{Prozessmodellierung}
  Gegeben sei folgende Modellbeschreibung:
  \par Im Präsidium der Universität einer norddeutschen Millionenstadt wird überlegt, die Arbeitsmoral der wissenschaftlichen Mitarbeiter bzw. Mitarbeiterinnen und der Studierenden zu verbessern, indem ein universitätseigener Radiosender betrieben wird. Vor allem soll durch ein attraktives Programm-Highlight am Morgen ein Anreiz fürs frühe Aufstehen geboten werden:
  \par Das Programm beginnt täglich um 5 Uhr mit Universitätsnachrichten und -neuigkeiten (mittlere Dauer 5 Minuten, Normalverteilung mit Standardabweichung 45 Sekunden), wobei kurz vorm Ende dieses Nachrichtenblocks von der Moderatorin eine schwierige Quizfrage vorgelesen wird, die auf dem Stoff eines Pflichtmoduls von einem der Bachelor-Studiengänge der Universität basiert.
  \par Wenn ein Radio-Hörer bzw. eine Radio-Hörerin die Antwort auf diese Frage weiß, kann er oder sie einen täglich wechselnden attraktiven Preis (z.B. einen Mensa- oder Bibliotheksgutschein) gewinnen. Hier heißt es aber, schnell zu sein: Wer die richtige Antwort zu wissen glaubt, muss zum Telefon greifen und die Nummer des Radiosenders wählen, was einschließlich des Aufbaus der Verbindung im Mittel 5 Sekunden dauert (Normalverteilung mit Standardabweichung 1 Sekunde).
  \begin{itemize}
    \item Wenn sich kein anderer Anrufer in der Telefonleitung des Radiosenders befindet, kommt ein Gespräch des Anrufers bzw. der Anruferin mit der Moderatorin der Radiosendung zustande. Das Gespräch dauert im Mittel 20 Sekunden (Normalverteilung mit Standardabweichung 5 Sekunden).
          \par Die Quizfrage ist so schwierig, dass Sie vorraussichtlich von nur 10\% der Anrufer richtig beantwortet wird. Im Fall einer richtigen Antwort ist das Quiz beendet.
          \par Der Anrufer bzw. die Anruferin erhält den Preis und niemand wird (für heute) mehr anrufen.
          \par Wenn die Frage falsch beantwortet wurde, haben andere Anrufer die Chance, die richtige Antwort zu geben und den Preis zu gewinnen.
    \item Wenn sich ein anderer Anrufer in der Telefonleitung des Radiosenders befindet, erhält der Anrufer bzw. die Anruferin ein "Besetzt"-Zeichen
          \par Er bzw. sie wird es nach jeweils einer Wartezeit von einer Sekunde (konstant) noch bis zu dreimal erneut versuchen, anzurufen, wobei wiederum jeweils eine Verbindung aufgebaut werden muss, siehe Angabe der Dauer oben. Falls dies immer wieder nur ein "Besetzt"-Zeichen ergibt oder das Quiz inzwischen beendet ist (siehe oben), ist er bzw. sie traurig und gibt die Teilnahme am heutigen Quiz auf.
  \end{itemize}
  \par Die statistischen Analysen der Marketing-Abteilung der Universität haben ergeben, dass vorraus ichtlich im Mittel 3 Sekunden nach Ende des Nachrichtenblocks (Exponentialverteilung) einem Hörer oder einer Hörerin eine Antwort-Idee einfällt und er bzw. sie zum Telefon greift, um seine bzw. ihre vermutete Lösung zu übermitteln, und dass jeweils im Abstand von wiederum im Mittel 3 Sekunden (Exponentialverteilung) immer ein weiterer Hörer bzw. eine weitere Hörerin auf die vermeintlich richtige Lösung kommt und anzurufen versucht, bis das Quiz durch richtige Antwort beendet wird.

  Bearbeiten Sie hierzu folgende Teilaufgaben.
  \begin{enumerate}
    \item\label{itm:prim2} Nennen Sie drei Beispiele für primär interessierende Leistungsgrößen, die eine Simulation des Modells untersuchen könnte.
    \item\label{itm:empf2} Würden Sie für die Durchführung einer solchen Simulationsstudie eine ereignis- oder prozessorientierte Modellierung empfehlen? Geben Sie zwei Gründe für Ihre Empfehlung an.
    \item\label{itm:ent2} Unabhängig von Ihrer Empfehlung in \ref{itm:empf2} soll im Rahmen der Simulationsstudie eine prozessorientierte Modellierung erstellt werden. Benennen Sie die zu modellierenden Prozesse und (sofern benötigt) Ressourcen.
    \item\label{itm:mod2} Erstellen Sie für jeden in \ref{itm:ent2} benannten Prozesstyp eine semi-formale Modellierung in Form eines BPMN-Kollaborationsdiagramms zur Simulation des beschriebenen Modells mit IYOPRO.
  \end{enumerate}

  \hint{
    \begin{itemize}
      \item Berücksichtigen Sie immer weitere Anrufe im Zeitverlauf, bis das Quiz beendet ist.
      \item Kennzeichnen Sie Verteilungen für die Dauer zeitkonsumierender Aktivitäten und die Länge von Zwischenankunftszeiten, eventuell benötigte Ressourcen sowie möglicherweise erforderliche Definitionen und Änderungen der Werte von prozesslokalen Attributen sowie globalen Variablen wie in der MuS-Vorlesung und -Übung durch Kommentare.
      \item BPMN-Start-/Zwischen-/Endereignisse können Sie in dieser Klausur mangels farbiger Stifte nicht in grün, gelb oder rot ausmalen. Verwenden Sie aber unbedingt, wie in BPMN üblich, eine einfache/dünne Umrandung für ein Startereignis, eine doppelte/dünne Umrandung für ein Zwischenereignis und eine einfache/dicke Umrandung für ein Endereignis.
      \item Verwenden Sie außerdem, wo immer dies sinnvoll ist, möglichst konkrete Typen von BPMN-Ereignissen, also z.B. Zeitsteuerung oder Nachrichtenereignisse statt allgemeinen ("unausgefüllten") Ereignissen.
      \item Wo die Modellbeschreibung nicht ausreichend präzise oder unvollständig ist, dürfen sinnvolle Annahmen getroffen werden. Nicht explizit in der Modellbeschreibung mit einer bestimmten Dauer erwähnte Aktivitäten dürfen als zeitverzugslos angenommen werden.
      \item Vergleichen Sie nach Abschluss der Aufgabe Ihre Antworten zu \ref{itm:ent2} und \ref{itm:mod2}. Insbesondere: Falls Sie während der Bearbeitung von \ref{itm:mod2} Korrekturbedarf bezüglich in der \ref{itm:ent2} benannten Entitäten, Ereignistypen oder Warteschlangen festgestellt haben, dann ändern Sie auch Ihre Antwort von \ref{itm:ent2} entsprechend!
    \end{itemize}
  }

  \begin{solution}
    \begin{enumerate}
      \item Primär interessierende Leistungsgrößen: Durchschnittliche Dauer des Quiz, Anzahl der Anrufe, Anzahl der Gewinner.
      \item Empfehlung: Prozessorientierte Modellierung. Gründe: Kontinuierliche Zeit, explizite Repräsentation von Ressourcen.
      \item Prozesse und Ressourcen: Prozesse: Nachrichtenblock, Quiz, Anruf. Ressourcen: Telefonleitung, Moderatorin.
      \item Modellierung: BPMN-Kollaborationsdiagramme für Prozesse
    \end{enumerate}
  \end{solution}
\end{exercise}

\sheet[2023]{Altklausur}
\begin{exercise}{Modellierung}
  Gegeben sei folgenden Modellbeschreibung:
  \par Stadion, wo Fußballspiele eines Vereins stattfinden und das logistische Herausforderungen für die Stadt mit sich bringt:
  \par Die Fans nehmen die S-Bahn vom Hbf der Stadt, welche sie zum Bahnhof in der Nähe des Stadions fährt. Alle 5 Minuten verlässt eine S-Bahn den Hbf mit 300-500 (gleichverteilung) Fans. Die Fahrtzeit ist normalverteilt x Minuten lang mit x Minuten Standardabweichung.
  \par Beim Bahnhof angekommen steigen die Fans aus, wobei der Ausstieg so beschrieben wird, dass konstant alle Zehntel Sekunden ein Fan aussteigt.
  \par Nun müssen die Fans noch einen Shuttle Bus nehmen, um zum Stadion zu gelangen. Um zum Shuttle Bus zu gelangen, müssen die Fans zu einem Tunnel laufen und diesen überqueren (begrenzte Anzahl darf durch), daher kann sich eine Warteschlange vor dem Tunnel bilden, bis alle durch sind. An der Bushaltestelle angekommen fährt regelmäßig alle 100 Sekunden ein Shuttle Bus los und befördert die Fans zum Stadion. Es wird angenommen, dass es keine Begrenzung der Anzahl an Fans gibt, die in den Bus einsteigen dürfen.
  \par Wenn das Warten auf dem Bus zu lange dauert und einen der jeweiligen Person spezifischen Maximalwert der Wartezeit überschritten wurde (exponentialverteilt im Mittel 300 Sekunden), bricht der Fan das Warten ab und macht sich zu Fuß auf dem Weg zum Stadion.

  Bearbeiten Sie hierzu folgende Teilaufgaben.
  \begin{enumerate}
    \item Nennen Sie drei Leistungsgrößen die bei einer Simulation dieses Modells analysiert werden könnten.
    \item\label{itm:empf3} Würden Sie eine ereignisorientierte oder eine prozessorientierte Modellierung vorschlagen?
    \item\label{itm:ent3} Nennen Sie Entitäten und teilen Sie diese in Prozesse \& Ressourcen ein.
    \item Unabhängig vom Ergebnis aus \ref{itm:empf3} soll eine prozessorientierte Modellierung mittels BPMN erfolgen. Prozessarten aus \ref{itm:ent3} modellieren.
  \end{enumerate}

  \begin{solution}
    \begin{enumerate}
      \item Leistungsgrößen: Durchschnittliche Wartezeit, Anzahl der Fans, Anzahl der Busse.
      \item Empfehlung: Ereignisorientierte Modellierung. Gründe: Diskrete Ereignisse, keine explizite Repräsentation von Ressourcen.
      \item Entitäten: Fans, S-Bahn, Bahnhof, Shuttle Bus, Tunnel. Prozesse: Fahrt mit der S-Bahn, Ausstieg am Bahnhof, Warten am Tunnel, Fahrt mit dem Shuttle Bus. Ressourcen: S-Bahn, Shuttle Bus, Tunnel.
      \item Modellierung: BPMN-Prozesse für Prozesse
    \end{enumerate}
  \end{solution}
\end{exercise}

\begin{exercise}{Mathematischer Themenblock}
  \begin{enumerate}
    \item Aufgabe mit zentralem Grenzwertsatz
    \item Ablaufsteuerung der ereignisorientierten Simulation beschreiben und dabei die Bedeutung der Ereignisliste erklären
    \item Unterschied zwischen ereignisbasierter Simulation und zeitintervallbasierter Simulation erklären
  \end{enumerate}

  \begin{solution}
    \begin{enumerate}
      \item Zentraler Grenzwertsatz: Aussage über die Verteilung der Summe von unabhängigen, identisch verteilten Zufallsvariablen.
      \item Ablaufsteuerung: Ereignisliste: Liste von Ereignissen, sortiert nach Zeitpunkt ihres Eintretens.
      \item Ereignisbasierte Simulation: Simulation, die auf diskreten Ereignissen basiert. Zeitintervallbasierte Simulation: Simulation, die auf kontinuierlicher Zeit basiert.
    \end{enumerate}
  \end{solution}
\end{exercise}

\begin{exercise}{Qualitative \& Quantitative Prozessanalysetechniken}
  \begin{enumerate}
    \item Nennen Sie jeweils zwei qualitative und zwei quantitative Prozessanalysetechniken und beschreiben Sie diese kurz.
    \item Wie kann man die Critical Task Execution optimieren?
  \end{enumerate}

  \begin{solution}
    \begin{enumerate}
      \item Qualitative Prozessanalysetechniken: Interviews, Beobachtungen. Quantitative Prozessanalysetechniken: Simulation, Prozessmining.
      \item Critical Task Execution optimieren: Durchführung von Critical Task Analysis, Identifikation von Engpässen, Optimierung der Engpässe.
    \end{enumerate}
  \end{solution}
\end{exercise}

\begin{exercise}{BPMN}
  \begin{enumerate}
    \item Nennen Sie zwei Nachteile von BPMN und beschreiben Sie diese kurz.
    \item Was ist Process Mining und wie kann man den Nachteilen von BPMN mithilfe von Process Mining entgegenwirken?
  \end{enumerate}

  \begin{solution}
    \begin{enumerate}
      \item Nachteile von BPMN: Komplexität, Abstraktion. Komplexität: BPMN-Diagramme können sehr komplex werden und dadurch unübersichtlich. Abstraktion: BPMN-Diagramme sind abstrakt und können dadurch von Nicht-Experten schwer verstanden werden.
      \item Process Mining: Analyse von Prozessdaten zur Identifikation, Überwachung und Verbesserung von Geschäftsprozessen. Process Mining kann die Nachteile von BPMN durch die Analyse von Prozessdaten und die Identifikation von Schwachstellen in Prozessen beheben.
    \end{enumerate}
  \end{solution}
\end{exercise}

\begin{exercise}{ERP-Systeme}
  \begin{enumerate}
    \item Würden Sie McDongerKing eine ERP-Standardsoftware oder Individualsoftware empfehlen? Begründen Sie Ihre Antwort.
    \item Wäre eine ERP-Cloud Lösung sinnvoll?
  \end{enumerate}

  \begin{solution}
    \begin{enumerate}
      \item ERP-Standardsoftware oder Individualsoftware: ERP-Standardsoftware: McDongerKing ist ein mittelständisches Unternehmen und hat keine speziellen Anforderungen an die Software. Individualsoftware: McDongerKing hat spezielle Anforderungen an die Software, die nicht durch Standardsoftware abgedeckt werden können.
      \item ERP-Cloud Lösung: Ja, eine ERP-Cloud Lösung wäre sinnvoll, da McDongerKing dadurch Kosten sparen und flexibel auf Änderungen reagieren kann.
    \end{enumerate}
  \end{solution}
\end{exercise}

\begin{exercise}{IT-Governance}
  Nennen Sie zwei Aufgaben der IT-Governance und beschreiben Sie diese kurz.

  \begin{solution}
    Aufgaben der IT-Governance: IT-Strategie: Entwicklung und Umsetzung einer IT-Strategie, IT-Compliance: Sicherstellung der Einhaltung von gesetzlichen und regulatorischen Anforderungen.
  \end{solution}
\end{exercise}

\begin{exercise}{IT-Strategie \& Digital Business Strategie}
  Erklären Sie den Unterschied zwischen IT-Strategie und Digital Business Strategie anhand von drei charakteristischen Eigenschaften der beiden und nennen Sie Beispiele.

  \begin{solution}
    IT-Strategie: Fokus auf IT-Infrastruktur, IT-Systeme, IT-Prozesse. Digital Business Strategie: Fokus auf digitale Geschäftsmodelle, digitale Produkte, digitale Services.
  \end{solution}
\end{exercise}

\begin{exercise}{Soziotechnische Systeme}
  Nennen Sie zwei Prinzipien der Entwicklung soziotechnischer Systeme und beschreiben Sie diese kurz.

  \begin{solution}
    Prinzipien der Entwicklung soziotechnischer Systeme: Partizipation: Einbeziehung der Mitarbeiter in den Entwicklungsprozess, Transparenz: Offenlegung von Entscheidungen und Prozessen.
  \end{solution}
\end{exercise}


\sheet[2022]{Altklausur}

\begin{exercise}{Ereignisorientierte Modellierung}
  \begin{enumerate}
    \item Nennen Sie vier typische Zwecke, warum man lieber an Modellen Untersuchungen durchführt, statt am Realsystem und je ein Beispiel nennen.
    \item Ereignisorientierte Modellierung: wie in Probeklausur.
  \end{enumerate}

  \begin{solution}
    \begin{enumerate}
      \item Zwecke von Modellen: Verständnis, Analyse, Prognose, Simulation. Beispiele: Verständnis: Modell eines Flughafens, Analyse: Modell eines Produktionsprozesses, Prognose: Modell des Klimawandels, Simulation: Modell eines Verkehrssystems.
      \item Ereignisorientierte Modellierung: wie in Probeklausur.
    \end{enumerate}
  \end{solution}
\end{exercise}

\begin{exercise}{Systemmodellierung}
  \begin{enumerate}
    \item Nennen Sie einen Vorteil und einen Nachteil für Systemmodellierung auf der Werkzeugebene und beschreiben Sie diese.
    \item Beschreiben Sie das Zwischenereignis einer empfangenen Nachricht.
  \end{enumerate}

  \begin{solution}
    \begin{enumerate}
      \item Vorteil: Strukturierung, Nachteil: Vereinfachung. Strukturierung: Systemmodellierung hilft, komplexe Systeme zu strukturieren und zu visualisieren. Vereinfachung: Systemmodelle sind abstrakt und können dadurch die Realität vereinfachen.
      \item Zwischenereignis einer empfangenen Nachricht: Ereignis, das auftritt, wenn eine Nachricht empfangen wird.
    \end{enumerate}
  \end{solution}
\end{exercise}

\begin{exercise}{Anlauf von Simulationen}
  Wann ist eine Simulation fertig "angelaufen"?

  \begin{solution}
    Eine Simulation ist fertig "angelaufen", wenn die Initialisierung abgeschlossen ist und die Simulation gestartet wurde.
  \end{solution}
\end{exercise}

\begin{exercise}{Teile von Organisationen}
  \begin{enumerate}
    \item Nennen Sie die fünf Teile einer Organisation und beschreiben Sie diese kurz.
    \item Ordnen Sie McDongerKing in ein Organisationsdesign ein und begründen Sie dies mit Mechanismen, Koordination und Intention.
  \end{enumerate}

  \begin{solution}
    \begin{enumerate}
      \item Teile einer Organisation: Aufbauorganisation, Ablauforganisation, Personalorganisation, Informationsorganisation, Entscheidungsorganisation. Aufbauorganisation: Struktur der Organisation, Ablauforganisation: Prozesse und Abläufe, Personalorganisation: Personalmanagement, Informationsorganisation: Informationsfluss, Entscheidungsorganisation: Entscheidungsprozesse.
      \item Organisationsdesign: Funktionale Organisation. Begründung: Mechanismen: Hierarchie, Koordination: Standardisierung, Intention: Effizienz.
    \end{enumerate}
  \end{solution}
\end{exercise}

\begin{exercise}{McDongerKing}
  Was muss man beachten, wenn McDongerKing DIU (Digitale Innovationseinheiten) einführen will?

  \begin{solution}
    Bei der Einführung von DIU muss McDongerKing darauf achten, dass die DIU in die bestehende Organisation integriert werden und die Mitarbeiter in den Entwicklungsprozess einbezogen werden.
  \end{solution}
\end{exercise}

\begin{exercise}{Digital Business Strategy}
  Definieren Sie Digital Business Strategy (DBS).

  \begin{solution}
    Digital Business Strategy (DBS) ist die Entwicklung und Umsetzung einer Strategie, um digitale Technologien und Geschäftsmodelle zu nutzen, um Wettbewerbsvorteile zu erzielen.
  \end{solution}
\end{exercise}

\end{document}