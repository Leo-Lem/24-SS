
% 2023 Erst
% Aufgabe 1
% Gegebene Modellbeschreibung:
% Stadion wo Fußballspiele eines Vereins stattfinden und das logistische Herausforderungen für die Stadt mit sich bringt. Die Fans nehmen die S-Bahn vom Hbf der Stadt welche sie zum Bahnhof in der Nähe des Stadions fährt.
% Alle 5 Minuten verlässt eine S-Bahn den Hbf mit 300-500 (gleichverteilung) Fans. Die Fahrtzeit ist normalverteilt x Minuten lang mit x Minuten Standardabweichung.
% Beim Bahnhof angekommen steigen die Fans aus wobei der Ausstieg so beschrieben wird, das konstant alle Zehntel Sekunden ein Fan aussteigt. Nun müssen die Fans noch einen Shuttle Bus nehmen um zum Stadion zu gelangen. Um zum Shuttle Bus zu gelangen, müssen die Fans zu einem Tunnel laufen und diesen überqueren (begrenzte Anzahl darf durch) daher kann sich eine Warteschlange vor dem Tunnel bilden, bis alle durch sind. An der Bushaltestelle angekommen fährt regelmäßig alle 100 Sekunden ein Shuttle Bus los und befördert die Fans zum Stadion. Es wird angenommen, dass es keine Begrenzung der Anzahl an Fans gibt, die in den Bus einsteigen dürfen. Wenn das warten auf dem bus zu lange dauert und einen der jeweiligen Person spezifischen Maximalwert der Wartezeit überschritten wurde (exponentialverteilt im Mittel 300 Sekunden), bricht der Fan das warten ab und macht sich zu Fuß auf dem Weg zum Stadion.
% (a) Nennen Sie drei Leistungsgrößen die bei einer Simulation dieses Modells analysiert werden könnten.
% (b) Würden Sie eine ereignisorientierte oder eine prozessorientierte Modellierung vorschlagen?
% (c) Entitäten nennen und in Prozesse und Ressourcen aufteilen.
% (d) Unabhängig vom Ergebnis aus (b) soll eine prozessorientierte Modellierung mittels
% BPMN erfolgen. Prozessarten aus c) modellieren.
% - Mathematischer Themenblock: Aufgabe mit zentralen Grenzwertsatz
% - Ablaufsteuerung der ereignisorientierten Simulation beschreiben und dabei die Bedeutung der Ereignisliste erklären
% - Unterschied zwischen ereignisbasierter Simulation und zeitintervallbasierter Simulation erklären
% ISO-Teil
% Aufgabe 1 - Co-Create Single Choice
% 10 Wahr/Falsch Aussagen gegeben und jeweiliges ankreuzen.
% Aufgabe x
% Qualitative und Quantitative Prozessanalysetechniken nennen und kurz beschreiben und sagen wie man CTE optimieren kann.
% Aufgabe 2
% Zwei Nachteile von BPMN beschreiben, Process Mining erklären und erläutern, wie man den Nachteilen mithilfe von Process Mining entgegenwirken kann.
% Aufgabe 3 - Informationssysteme
% ERP-Standardsoftware oder Individualsoftware für McDonger King aus der Übung. Und zusätzlich angeben, ob ERP-Cloud Lösung sinnvoll wäre.
% Aufgabe 3
% zwei Aufgaben der IT-Governance nennen und beschreiben.
% Aufgabe 4
% Unterschied zwischen IT-Strategie und Digital Business Strategie anhand von drei charakteristischen Eigenschaften der beiden erklären und Beispiele nennen.
% Aufgabe 5
% Rolle und Aufgaben der CDO erklären und einem Zitat von Hausaufgabe 3 entweder zustimmen oder ablehnen. Das Zitat besagt, dass nach dem die Digitalisierung überwunden wurde, es auch keine CDO‘s mehr geben wird bzw. diese nicht mehr gebraucht werden würden.
% Aufgabe x
% Zwei Prinzipien der Entwicklung soziotechnischen Systeme nennen und beschreiben

% 2022 Erst?
% MUS:
% Aufgabe 1
% 4 typische Zwecke nennen, warum man lieber an Modellen Untersuchungen durchführt, statt am Realsystem und je ein Beispiel nennen.
% Ereignisorientierte modellieren (exakte aufgabenstellung wie in der Beispielklausur):
% es gibt in einer uni einen getränkeautomaten, mit 4 sorten. studenten kommen alle 30 min zum automaten (exponentialverteilt). dabei wird geld in den automaten geworfen, das den preis des getränks auf jeden fall deckt.
% bei der getränkeauswahl gibt es zwei möglichkeiten
% 1. dasgetränkistverfügbardanngibtderautomatdasgetränkrausund
% gibt ggf. restgeld zurück.
% 2. dasgetränkistnichtmehrda.danngibteseine50%chance,dasssich
% der student für sein zweitliebstes getränk entscheidet. ist das auch nicht da zieht geht er durstig weg.
% der gesammte prozess von bestellen kann als atomar betrachtet werden.
% alle 7 tage(konstant) kommt ein getränkelieferant und füllt alle getränke wieder auf.
% da bauarbeiten herrschen kommt es ab und zu (alle 120min) exponentialverteilt zum stromausfall. dieser dauert (2 stunden) exponentialverteilt an. es dauert dann wieder 120 min exponentialverteilt, bis der automat wieder vom ausfall betroffen ist. das für nachfüllen des automaten spielt der ausfall keine rolle

%  Aufgabe 2
% a) Einen Vorteil und einen Nachteil für Systemmodellierung auf der Werkzeugebene nennen und beschreiben.
% b) das zwischenereignis einer empfangenen nachricht bescheiben. (ziemlich detailiert mit semantik, beziehung...)
% Aufgabe 3)
% Etwas zum Anlauf von Simulationen, wann ist eine Simulation fertig “angelaufen?”
% ISO
% Aufgabe 1)
% multiple choice aus "co-create your exam"
% Aufgabe 2,
% a)
% Die 5 Teile einer Organisation benennen und kurz beschreiben
% b)
% einordnen in welches organisationsdesign Mc donger king fällt und mit mechanismen, koordination, intention begründen.
% Aufgabe 3)
% Aufgabe zur Gastvorlesung
% a) Was ist ein Datenkatalog und wozu wird er benutzt? b) Wie verändert er die Führungseben/-kultur?
% Aufgabe 4)
% “Mit dem, was wir in den übungen gemacht haben Mc Donger King und Ambidextrieforschung in kontext setzen.”

% “Was muss man beachten, wenn mc donger king DIU (Digitale Innovations Einheiten) einführen will?”
% Aufgabe 5)
% Definition von Digital Business Strategy (DBS)